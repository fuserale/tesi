% !TEX root = ../main.tex

% ---------------------------------------------------------------
% Introdaction
% ---------------------------------------------------------------


\chapter{Introduzione}\label{cap1:Introduzione}

% ---------------------------------------------------------------
% Introdaction
% -Motivation
% ---------------------------------------------------------------



% ---------------------------------------------------------------
% Introdaction
% -Motivation
% -Methodology
% -Thesis Contribution
% ---------------------------------------------------------------
\section{Contributo della Tesi}\label{cap1:Contributo della Tesi}
In questa lavoro di tesi, si è cercati di sviluppare una metodologia non supervisionata, ossia che non abbia bisogno dell'intervento dell'essere umano, per identificare e prevenire gli episodi di FoG. Questo è un problema molto debilitante per i malati di Parkinson che, allo studio attuale, non ha una soluzione efficace dal punto di vista medico e di cui non si conoscono a fondo le cause.\\
Per sviluppare tale metodologia, è stato utilizzato un dataset in cui sono presenti 10 pazienti, tutti affetti dal morbo di Parkinson, ed 8 di loro presentano episodi di FoG. I contesti di FoG sono stati identificati dal dottore tramite l'utilizzo di videoregistrazioni ed i dati sono stati raccolti attraverso 3 accellerometri, posizionati sulla caviglia, sul ginocchio e sulla zona lombare del paziente.\\
Il primo passo è stato quello di dividere in finestre temporali e di filtrare i dati ottenuti dagli accelerometri, al fine di calcolare sulle stesse delle grandezze su cui sono state applicate gli algoritmi di clustering, al fine di dividere in automatico le occorrenze di FoG, di pre-FoG e di camminata normale. E' stato poi condotto uno studio sulla possibilità di utilizzare un solo accellerometro invece di 3 contemporaneamente. Poichè il dividere in finestre temporali fissate poteva portare all'eliminazione di contesti interessati, i.e. episodi di FoG, è stato condotto uno studio anche su finestre temporali dinamiche, ossia in base all'etichettatura data dal dottore nel dataset. Infine, è stato usato un approccio di discriminazione delle caratteristiche per dividere maggiormente i vari contenti.

% ---------------------------------------------------------------
% Introdaction
% -Motivation
% -Methodology
% -Thesis Contribution
% -Outline
% ---------------------------------------------------------------

\section{Struttura della Tesi}\label{cap1:Struttura della Tesi}
Il resto della tesi è organizzata nel modo seguente:
\begin{itemize}
	\item Nel \textbf{\chapterref{chap2:related}} si presentano i risultati dei principali lavori svolti sul FoG;
	\item  Nel \textbf{\chapterref{chap3:background}} si presentano a livello teorico le problematiche del FoG e gli strumenti su cui si basa il lavoro svolto;
	\item Nel \textbf{\chapterref{chap4:Goals}} si presenta qual'è l'obiettivo finale del lavoro che viene svolto;
	\item Nel \textbf{\chapterref{chap5:Automatic}} si presenta il lavoro svolto;
	\item Nel \textbf{\chapterref{chap6:SW}} si fornisce l'implementazione software del lavoro;
	\item Nel \textbf{\chapterref{chap9:Concl}} si riassumo i risultati ottenuti e si presentano possibili miglioramenti al lavoro svolto.
\end{itemize}
 