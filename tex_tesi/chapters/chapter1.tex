% !TEX root = ../main.tex

% ---------------------------------------------------------------
% Introdaction
% ---------------------------------------------------------------


\chapter{Introduzione}\label{cap1:Introduzione}

% ---------------------------------------------------------------
% Introdaction
% -Motivation
% ---------------------------------------------------------------
La Malattia di Parkinson è una patologia neurodegenerativa che coinvolge in maniera elettiva la capacità di programmare ed eseguire il movimento, senza risparmiare altri aspetti dell’individuo come la sfera cognitiva e comportamentale. Questi aspetti, unitamente al decorso cronico e progressivo della malattia, determinano una compromissione delle attività di vita quotidiana e delle relazioni interpersonali.

Tra i sintomi della malattia di Parkinson, il Freezing of Gait (FOG) può sicuramente essere considerato uno dei più debilitanti. Il Freezing nella malattia di Parkinson, detto anche congelamento o semplicemente blocco motorio, è un’improvvisa, temporanea e involontaria incapacità di iniziare un movimento. È un disturbo che insorge nel corso dell’evoluzione della malattia di cui costituisce un sintomo indipendente e generalmente resistente al trattamento con levodopa. Tale fenomeno si può verificare in ogni momento e i pazienti che lo sperimentano affermano che: \textit{«è come se i piedi rimanessero, per qualche istante, incollati al suolo con la conseguente impossibilità di eseguire il passo successivo»}. In realtà, il Freezing si può verificare anche durante azioni differenti dal cammino come ad esempio l’alzarsi da una sedia o il raccogliere un oggetto. Alcune persone sono più predisposte di altre a subire episodi di congelamento. Tali episodi, si possono verificare sia quando il soggetto è in astinenza da farmaci dopaminergici, in questo caso si parla di “Freezing off”, sia quando il soggetto sta assumendo i farmaci, “Freezing on”. \\

% ---------------------------------------------------------------
% Introdaction
% -Motivation
% -Methodology
% -Thesis Contribution
% ---------------------------------------------------------------
\section{Contributo della Tesi}\label{cap1:Contributo della Tesi}
In questa lavoro di tesi, si è cercati di sviluppare una metodologia di identificazione degli episodi di FoG e preFoG, ossia una fase transitoria tra un'attività normale ed un episodio di Freezing. Questo è un problema molto debilitante per i malati di Parkinson che, allo studio attuale, non ha una soluzione efficace dal punto di vista medico e di cui non si conoscono a fondo le cause.\\
Per sviluppare tale metodologia, è stato utilizzato un dataset in cui sono presenti 10 pazienti, tutti affetti dal morbo di Parkinson, ed 8 di loro presentano episodi di FoG. I contesti di FoG sono stati identificati dal dottore tramite l'utilizzo di videoregistrazioni ed i dati sono stati raccolti attraverso 3 accelerometri, posizionati sulla caviglia, sul ginocchio e sulla zona lombare del paziente.\\
Il primo passo è stato quello di verificare che può esistere una distinzione tra le 3 classi del nostro dataset, ossia attività normale, preFoG e FoG tramite l'utilizzo di un Linear Discriminant Analysis (LDA). Una volta verificato, e' stato condotto uno studio sugli intervalli con cui dividere i dati del dataset, cioè quanto deve essere grande la finestra temporale su cui calcolo le feature dai dati grezzi e quanta sovrapposizione è opportuno avere. Questo studio è stato condotto sia usando LDA che usando un approccio di clustering, che permette di etichettare i dati senza sapere a quale classe appartengano, al fine di cercare di sostituire il dottore nella prima fase di test. Infine, si accenna ad un primo approccio di classificazione per la previsione di classi su nuovi dati.

% ---------------------------------------------------------------
% Introdaction
% -Motivation
% -Methodology
% -Thesis Contribution
% -Outline
% ---------------------------------------------------------------

Il resto della tesi è organizzata nel modo seguente: nel \textbf{\chapterref{chap2:related}} si presentano i risultati dei principali lavori svolti sul FoG; nel \textbf{\chapterref{chap3:background}} si presentano a livello teorico le problematiche del FoG e gli strumenti su cui si basa il lavoro svolto; nel \textbf{\chapterref{chap4:Goals}} si presenta qual'è l'obiettivo finale del lavoro che viene svolto; nel \textbf{\chapterref{chap5:Automatic}} si presenta il lavoro svolto; nel \textbf{\chapterref{chap9:Concl}} si riassumo i risultati ottenuti e si presentano possibili miglioramenti al lavoro svolto.
 