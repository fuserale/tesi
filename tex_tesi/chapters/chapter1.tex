% !TEX root = ../main.tex

% ---------------------------------------------------------------
% Introdaction
% ---------------------------------------------------------------


\chapter{Introduzione}\label{cap1:Introduzione}

% ---------------------------------------------------------------
% Introdaction
% -Motivation
% ---------------------------------------------------------------
La Malattia di Parkinson è una patologia neurodegenerativa che coinvolge in maniera elettiva la capacità di programmare ed eseguire il movimento, senza risparmiare altri aspetti dell’individuo come la sfera cognitiva e comportamentale. Questi aspetti, unitamente al decorso cronico e progressivo della malattia, determinano una compromissione delle attività di vita quotidiana e delle relazioni interpersonali.

Tra i sintomi della malattia di Parkinson, il Freezing of Gait (FOG) può sicuramente essere considerato uno dei più debilitanti. Il Freezing nella malattia di Parkinson, detto anche congelamento o semplicemente blocco motorio, è un’improvvisa, temporanea e involontaria incapacità di iniziare o proseguire un movimento. È un disturbo che insorge nel corso dell’evoluzione della malattia di cui costituisce un sintomo indipendente e generalmente resistente al trattamento con levodopa. Tale fenomeno si può verificare in ogni momento e i pazienti che lo sperimentano affermano che: \textit{«è come se i piedi rimanessero, per qualche istante, incollati al suolo con la conseguente impossibilità di eseguire il passo successivo»}. In realtà, il Freezing si può verificare anche durante azioni differenti dal cammino come ad esempio l’alzarsi da una sedia o il raccogliere un oggetto. Alcune persone sono più predisposte di altre a subire episodi di congelamento. Tali episodi si possono verificare sia quando il soggetto è in astinenza da farmaci dopaminergici, in questo caso si parla di “Freezing off”, sia quando il soggetto sta assumendo i farmaci, “Freezing on”. \\

% ---------------------------------------------------------------
% Introdaction
% -Motivation
% -Methodology
% -Thesis Contribution
% ---------------------------------------------------------------
\section{Contributo della Tesi}\label{cap1:Contributo della Tesi}
Il FOG é un problema molto debilitante per i malati di Parkinson che, allo studio attuale, non ha una soluzione efficace né dal punto di vista medico né da quello informatico. In questa lavoro di tesi, si è cercato di sviluppare una metodologia, basata su dati provenienti da accelerometri studiati mediante algoritmi, di identificazione degli episodi di FoG e preFoG, ossia una fase transitoria tra un'attività normale ed un episodio di Freezing.\\
Il lavoro condotto, partendo da dati raccolti dagli assi x,y,z di più accelerometri, verifica la presenza di una nuova tipologia di classe, che chiameremo preFOG, dei dati e conduce uno studio algoritmico non supervisionato sulla divisione dei dati stessi, al fine di identificare tale classe in un contesto real-time e fornire uno stimolo uditorio, mediante dei dispositivi indossabili, al paziente, per evitare l'occorrenza del Freezing.

% ---------------------------------------------------------------
% Introdaction
% -Motivation
% -Methodology
% -Thesis Contribution
% -Outline
% ---------------------------------------------------------------

Il resto della tesi è organizzata nel modo seguente: nel \textbf{\chapterref{chap2:related}} si presentano i risultati dei principali lavori svolti sul FoG; nel \textbf{\chapterref{chap3:background}} si presentano a livello teorico le problematiche del FoG e gli strumenti su cui si basa il lavoro svolto; nel \textbf{\chapterref{chap4:Goals}} si presenta qual'è l'obiettivo finale del lavoro che viene svolto; nel \textbf{\chapterref{chap5:Automatic}} si presenta il lavoro svolto; nel \textbf{\chapterref{chap9:Concl}} si riassumono i risultati ottenuti e si presentano possibili miglioramenti al lavoro svolto.
 