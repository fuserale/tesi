% !TEX root = ../main.tex
% ---------------------------------------------------------------
% GOALS
% ---------------------------------------------------------------



\chapter{Motivazioni ed obiettivi}\label{chap4:Goals}
Quello del FoG, come introdotto in precedenza, è un problema che debilita in modo anche grave il paziente affetto da Parkinson. L'obiettivo della nostra metodologia è quello di riconoscere le situazioni di FoG, attraverso l'identificazione di pattern o condizioni che portano il paziente a bloccarsi ed incorrere nel rischio di capire. A tale scopo, è necessario sviluppare una metodologia che permetta di identificare tali avvenimenti, quindi fornisca, nel modo più accurato possibile, l'informazione sulla presenza o meno di FoG, per poi definire anche gli istanti immediatamente precedenti al blocco, che chiamiamo preFoG. Una volta che si riconosco tali intervalli, tramite uno stimolo uditorio dato nella fase di preFoG si può evitare l'occorrenza del FoG e quindi permette al paziente di continuare nell'attività che stava svolgendo senza che abbia blocchi o rischi di cadere.\\
\begin{tikzpicture}
\node [mybox] (box){%
	\begin{minipage}{.96\textwidth}
		Quello che, dunque, si vuole progettare è un algoritmo automatizzato non supervisionato che, attraverso l'uso di accelerometri posizionati sul paziente, permetta di identificare gli istanti di pre-FoG e di FoG del paziente. Questo permetterebbe di sostuire il dottore, o fisioterapista, nell'identificazione di episodi di FoG nella  fase di test.
	\end{minipage}
};
\end{tikzpicture}%

Infatti, fino ad adesso, al paziente venivano fatti fare dei percorsi ed il dottore, tramite l'utilizzo di registrazioni, registrava i FoG ed etichettava dunque tutti gli istanti della registrazione in basse all'occorrenza del blocco oppure no. Con questo approccio, invece, tutto questo verrebbe effettuato dall'algoritmo, senza il bisogno dell'intervento umano.\\
