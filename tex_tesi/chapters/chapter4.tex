% !TEX root = ../main.tex
% ---------------------------------------------------------------
% GOALS
% ---------------------------------------------------------------



\chapter{Motivazioni ed obiettivi}\label{chap4:Goals}
%Quello del FoG, come introdotto in precedenza, è un problema che debilita in modo anche grave il paziente affetto da Parkinson. L'obiettivo della nostra metodologia è quello di riconoscere le situazioni di FoG e preFoG. A tale scopo, è necessario sviluppare una metodologia che permetta di identificare tali avvenimenti, quindi fornisca, nel modo più accurato possibile, l'informazione sulla presenza o meno di FoG ed anche gli istanti immediatamente precedenti al blocco, che chiamiamo preFoG. \\
%\begin{tikzpicture}
%\node [mybox] (box){%
%	\begin{minipage}{.96\textwidth}
%		Quello che, dunque, si vuole progettare è un algoritmo automatizzato non supervisionato che, attraverso l'uso di accelerometri posizionati sul paziente, permetta di identificare gli istanti di pre-FoG e di FoG del paziente evidenziando una distinzione tra le due classi.
%	\end{minipage}
%};
%\end{tikzpicture}%
%
%Se viene trovata tale distinzione, significa che ci si può concentrare sulle occorrenze di preFoG al fine di anticipare gli episodi di FoG e quindi permette al paziente di evitare il più possibile il succedere di tale problema.
