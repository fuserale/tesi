% !TEX root = ../main.tex

\cleardoublepage
\thispagestyle{empty}

\leavevmode \\[0.86cm]
\begin{center}
\rule{\textwidth}{.4pt} \\
\end{center}
{\LARGE\textbf{Abstract}}
\vspace{1cm}

Freezing of gait (FOG) is defined as a brief, episodic absence or marked reduction of forward progression of the feet despite the intention to walk. It is one of the most debilitating motor symptoms in patients with Parkinson's disease (PD) as it may lead to falls and a loss of independence. The pathophysiology of FOG seems to differ from the cardinal features of PD and is still largely unknown. In this paper, we focus on demonstrate the existence of some movements that could lead to FoG and we include this movements in a new class that we call preFoG. We also have conducted an exhaustive study on the right choose of the interval and overlap to better divide our classes, noFoG, preFoG and FoG. We have made a first data classification approach using the preFoG class. At the same time, we conducted a study using clustering algorithms in order to try to replace the doctor in the first phase of data label generation. We have concluded that the best data division interval is 2 seconds window and 1 second overlap. For the classification we arrived at an accuracy of 90\% using data from more patients, while for the clustering phase at an accuracy of 70\%.

\clearpage
\thispagestyle{empty}