% !TEX root = ../main.tex

% ---------------------------------------------------------------
% RELATED WORKS
% ---------------------------------------------------------------

\chapter{Letteratura}\label{chap2:related}
Il problema del FOG è stato analizzato tramite una grande varietà di sistemi e sensori. Alcuni di questi, però, non sono utilizzabili durante la vita quotidiana dei pazienti poiché posso essere disponibili solo in ambienti di laboratorio. Esempi di questi sistemi sono le piattaforme di pressione\cite{38}, le quali sono non portatili, l'elettromiografia (EMG)\cite{25}, l'elettroencefalogramma (EEG)\cite{42} o la conduttanza della pelle\cite{43}, il quale comporta il piazzamento di elettrodi sulla pelle in aggiunta ad un sistema di rilevamento per raccogliere i dati.
Altri sistemi invasivi sono i goniometri a ginocchio\cite{23} o sistemi che fanno uso di camere e video, i quali hanno una bassa tolleranza del paziente in un'ambiente che non sia di laboratorio\cite{23,39,44}. Quindi, dato che il monitoraggio del PD dovrebbe essere deambulatorio e  durare diverse ore al fine di ricavare utili informazioni cliniche\cite{34,45}, la maggior parte dei lavori si è basata su sistemi non invasivi come i dispositivi indossabili basati su circuiti microelettromeccanici (MEMS). \newline
Nel 2003, Han et al. hanno usato MEMS basati su sistemi inerziali, come gli accelerometri, per esplorare le caratteristiche collegate agli episodi di FoG. Hanno trovato che la frequenza di cammino nei pazienti che indossavano gli accelerometri nella caviglia era intorno ai 6-8 Hz\cite{19}. Nel 2008, Moore et al. hanno proposto una metodologia per identificare FoG con un'accelerometro posizionato nella caviglia nella quale hanno descritto il Freezing Index (FI), ossia il quoziente del rapporto della Densità Spettrale di Potenza (PSD) tra 3 ed 8 Hz, chiamata Freezing Band (FB), con la PSD tra 0.5 e 3 Hz, denominata Walking Band (WB)\cite{21}. Quando il FI supera una certa soglia (Freezing Threshold (FTH)), si considera che si sia verificato un episodio di FoG. A causa della presenza dei falsi positivi (FP) quando il paziente è a riposo, Bachlin et al. hanno introdotto il concetto di Power Index (PI), definito come la somma della WB e FB, il quale viene comparato con la Power Threshold (PTH) al fine di stabilire se c'era una quantità rilevante di movimento nel momento in cui il FI era alto, ossia oltre la soglia\cite{21}. PI indica la quantità di movimento, perciò situazioni nelle quali il paziente non si stesse muovendo volontariamente sono state eliminate. In quest'ultima versione dell'algoritmo, quindi, un espisodio di FoG è occorso se FI>FTH e PI>PTH. Questo metodo è il più avanzato nel riconoscimento di FoG dato il suo scarso costo computazionale e le sue buone performance\cite{22}. \newline
L'algoritmo Moore-Bächlin FoG Algorithm (MBFA) è stato ampiamente utilizzato nell'analisi del FoG, anche se di solito in condizioni di laboratorio e molto spesso con pochi pazienti. Jovanov et al. hanno implementato un algoritmo real time, anche se un solo volontario è stato usato per testare l'algoritmo. Inoltre, nessun risultato su sensitività e specificità è stato riportato\cite{22}. Zabaleta et al. hanno analizzato il FoG per mezzo di accelerometri a tre assi e giroscopi a due assi in differenti locazioni degli arti inferiori. La caratteristica principale ad essere stata analizzata è il FI in congiunzione con i cambiamenti della densità spettrale di potenza. Sono stati capaci di identificare correttamente l'82.7\% degli episodi di FoG con i sensori inerziali posizionati su entrambe le caviglie, anche se in soli 2 pazienti\cite{24}.  \newline
Più recentemente, Niazmand et al. (2011) hanno presentato il Mimed-Pants\cite{26}, pantaloni da jogging lavabili con 5 accelerometri integrati. Hanno usato MBFA per identificare FoG, ottenendo un 88.3\% in sensitività e 85.3\% in specificità con 6 pazienti in brevi e controllati test focalizzati nell'indurre FoG senza tenere conto dei FP. Nel 2012, Zhao et al.\cite{46} hanno sviluppato un algoritmo embedded basato sull'approccio MBFA all'interno del sistema Mimed-Pants ottenendo un 81\% in sensitività con 8 pazienti usando dei test simili ai precedenti. Più recentemente, Mazilu et al. hanno proposto un nuovo algoritmo online usando 3 accelerometri e comparando diversi classificatori di machine learning che sfruttavano le caratteristiche del MBFA, aggiungendone di nuovi, in 10 pazienti\cite{48}. I risultati ottenuti sono stati migliori rispetto ai precedenti, con un 95\% per specificità (abilità nell'identificare i FoG) e sensitività (abilità nell'identificare i noFoG), con differenti classificatori. Questi test, però, sono stati condotti in situazioni di controllo e, inoltre, la metodologia di validazione sovrastimava le prestazioni delle misure poiché i classificatori erano allenati, iterativamente, con tutte le finestre del segnale disponibili da un paziente escludendone una, la quale veniva usata per ottenere le prestazioni citate. Inoltre, le sequenze di allenamento e di test erano molto simili, il che è molto diverso da normali situazioni. Quindi, ci si aspetta che le riportate specificità e sensitività calino drasticamente in situazioni non controllate. Sempre Mazilu et al, hanno ipotizzato l'esistenza di una terza classe, da loro chiamata preFOG, che si posizionerebbe prima delle occorrenze di FoG e racchiude tutti i movimenti che portano al Freezing. Tentando un approccio di cross-validation con questa nuova classe, é stata raggiunta una misura del 70\% per quanto rigurda la F1-measure (media armonica tra precisione e sensitività).\cite{12} \newline
Nel 2013, Moore et al. hanno pubblicato il più recente lavoro focalizzato sul MBFA. In questo, hanno confrontato differenti configurazioni applicando lo stesso algoritmo in 25 pazienti, dei quali 20 hanno avuto episodi di FoG. Diversi posizionamenti dei sensori e valori per PTH e FTH sono stati valutati al fine di trovare le condizioni ottimali. I risultati migliori sono stati ottenuti con soglie molto elevate, anche se con queste Moore et al. hanno riportato una rilevante perdita di sensitività negli episodi brevi che, paradossalmente, sono quelli più frequenti nei pazienti affetti da PD\cite{27}. In un test più complesso eseguito precedentemente\cite{20} usando fino a 7 sensori ed un protocollo di test più lungo, sono stati ottenuti una sensitività e specificità sopra al 70\%, anche se, in certe configurazioni (sistema installato nella zona lombare), sia per la sensitività che la specificità hanno raggiunto valori oltre l'80\%. In un approccio differente, Tripoli et al. hanno testato diverse configurazioni e locazioni dei sensori al fine di trovare la migliore configurazione\cite{52}. Il lavoro è stato svolto con 5 pazienti ed in condizioni controllate, usando uno specifico protocollo progettato per stimolare il FoG e senza test di FP. In tale lavoro, hanno integrato 2 giroscopi oltre a 6 accelerometri posizionati in posizioni differenti del corpo. Con tutti i sensori indossati, è stata ottenuta un'accuratezza del 96.11\%, una specificità del 98.74\% e, eseguendo i test su tutti i pazienti tranne uno, una sensitività dell'81.94\%. D'altra parte, con una IMU singola nella zona lombare hanno riportato una sensitività del 75\% ed una specificità del 95\%, anche se l'algoritmo non è stato confrontato con nessun altro metodo usato sotto le stesse condizioni. \newline
Mazilu et al.\cite{50} hanno investigato un approccio di apprendimento non supervisionato per costruire un input ottimale per un classificatore ad albero di decisione con il dataset del progetto DAPHNET (10 pazienti PD). Il loro approccio è stato comparato ad un analogo basato su MBFA nel quale il FI e l'energia della banda spettrale tra 0.5 Hz e 8 Hz sono state valutate. L'allenamento ed i test erano dipendenti dall'utilizzatore e sotto condizioni controllate. I risultati superano l'approccio MBFA similare dell'8.1\% in termini di punteggio dell'F1. Un altro approccio è stato presentato da Rodriguez et al., i quali hanno proposto un metodo per contestualizzare gli episodi di FoG tramite un algoritmo di riconoscimento dell'attività, il quale rifiutava i FP quando il paziente era seduto o eseguiva attività quali disegnare o digitare in un laptop. La specificità è stata aumentata in media del 5\%, arrivando anche ad miglioramento dell' 11.9\% in certi casi\cite{50}. Il metodo che aggiungeva la contestualizzazione, però, non ha contribuito a migliorare la sensitività. Altri studi hanno analizzato la variabilità della camminata tra un episodio di FoG e condizioni normali. Anche se i risultati sono interessanti, hanno fallito nell'includere i falsi positivi ed un'affidabile classificazione non è stata eseguita\cite{53,54}. Un paper recente di Zach et al. presenta una nuova metodologia per suscitare FoG in condizioni di laboratorio controllate, le quali sono state valutate con l'algoritmo MBFA ottenendo una sensitività del 75\% ed una specificità del 76\%\cite{31}. \newline
Infine, Alrichs et al., all'interno del progetto REMPARK\cite{55}, usano una Support Vector Machine (SVM) per rilevare episodi di FoG in 8 pazienti con PD in ambiente casalingo. Il metodo include test in differenti prove motorie usando un singolo accelerometro nella zona lombare, raggiungendo un'accuratezza del 90\%. La specificità, però, è stata calcolata solo con pazienti non FoG, il che può portare a predizioni non affidabili in quanto il modello non è stato testato con pazienti PD con FoG, i quali hanno movimenti molto diversi dai pazienti che non soffrono di FoG. Inoltre, la valutazione è stata eseguita su finestre di un minuto, tempo che è considerato troppo lungo per un'implementazione online\cite{28}. Sempre all'interno del progetto REMPARK, Rodriguez et al. hanno presentato un lavoro che utilizza un algoritmo per rilevare FoG tramite un approccio di machine learning basato su SVM ed un singolo accelerometro a 3 assi indossato nella zona lombare\cite{HD}.Il metodo è stato valutato su 21 pazienti affetti da PD in ambienti casalinghi sotto due condizioni: un modello generico testato su tutti i pazienti tranne uno ed un secondo modello personalizzato sull'utente che usa parte del dataset del paziente stesso. I risultati mostrano un significativo vantaggio del modello personalizzato rispetto a quello generico, portando ad un miglioramento in media, sia della sensitività che della specificità, del 7.2\%. Inoltre, l'approccio adottato è stato comparato con i metodi più utilizzati per la detenzione del FoG basati sull'algoritmo MBFA. I risultati del metodo generico mostrano un miglioramento in media dell'11.2\% rispetto a metodi MBFA generici, mentre quello personalizzato porta ad un miglioramento del 10\% rispetto ad altri metodi specifici sul paziente.\\
In tabella \ref{letteratura} si riassume quanto detto in questa sezione.
% Please add the following required packages to your document preamble:
% \usepackage{graphicx}
\begin{table}[]
	\centering
	\caption{Tabella riassuntiva delle metodologie proposte}
	\label{letteratura}
	\resizebox{\textwidth}{!}{%
		\begin{tabular}{|l|l|l|p{4cm}|}
			\hline
			\textbf{Lavoro}  & \textbf{Anno} & \textbf{Metodo algoritmico}               & \textbf{Risultati più significanti presentati}                                                         \\ \hline
			Han et al.       & 2003          & Test Statistici                           & Osservazione della frequenza tra 6-8Hz in un episodio di FOG                                           \\ \hline
			Moore et al.     & 2008          & Basato su soglie                          & Soglia generale: 78.3\% di successo. Soglia personalizzata: 89.1\% di successo                         \\ \hline
			Zabaleta et al.  & 2008          & Classificatore Lineare                    & 82,7\% successo                                                                                        \\ \hline
			Jovanov et al.   & 2009          & Basato su soglie                          & Algoritmo real time con veloce risposta. Risultati non riportati                                       \\ \hline
			Niazmand et al.  & 2011          & Basato su soglie                          & 88,3\% sensitività e 85,3\% specificità                                                                \\ \hline
			Zhao et al.      & 2012          & Basato su soglie                          & 81,7\% sensitività                                                                                     \\ \hline
			Mazilu et al.    & 2012          & Differenti classificatori                 & 95\% sensitività e 95\% specificità con certe configurazioni                                           \\ \hline
			Mazilu et al.    & 2013          & Albero di decisione                       & 70\% F1-measure per il preFOG                                                                          \\ \hline
			Moore et al.     & 2013          & Basato su soglie                          & sensitività e specificità \textgreater70\%                                                             \\ \hline
			Tripoli et al.   & 2013          & Differenti classificatori usando entropia & 96,11\% accuratezza con 6 accelerometri e 2 giroscopi                                                  \\ \hline
			Rodriguez et al. & 2014          & Basato su soglie                          & Miglioramento del 5\% rispetto a Moore nella specificitŕ grazie alla contestualizzazione della postura \\ \hline
			Alrichs et al.   & 2015          & Support Vector Machine                    & 90\% accuratezza                                                                                       \\ \hline
			Rodriguez et al. & 2017          & Support Vector Machine                    & Miglioramento dell'11\% in media rispetto a Moore                                                      \\ \hline
		\end{tabular}%
	}
\end{table}