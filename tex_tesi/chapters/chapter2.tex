% !TEX root = ../main.tex

% ---------------------------------------------------------------
% RELATED WORKS
% ---------------------------------------------------------------

\chapter{Letteratura}\label{chap2:related}
Il problema del FOG è stato analizzato tramite una grande varietà di sistemi e sensori. Alcuni di questi, però, non sono utilizzabili durante la vita quotidiana dei pazienti poiché posso essere disponibili solo in ambienti di laboratorio. Esempi di questi sistemi sono le piattaforme di pressione\cite{38}, le quali sono non portatili, l'elettromiografia (EMG)\cite{25}, l'elettroencefalogramma (EEG)\cite{42} o la conduttanza della pelle\cite{43}, il quale comporta il pizzamento di elettrodi sulla pelle in aggiunta ad sistema di rilevamento per raccogliere i dati.
Altri sistemi invasivi sono i goniometri a ginocchio\cite{23} o sistemi che fanno uso di camere e video, i quali hanno una bassa tolleranza del paziente in un'ambiente che non sia di laboratorio\cite{23,39,44}. Quindi, dato che il monitoraggio del PD dovrebbe essere deambulatorio e  durare diverse ore al fine di ricavare utili informazioni cliniche\cite{34,45}, la maggior parte dei lavori si è basata su sistemi non invasivi come i dispositivi indossabili basati su circuiti microelettromeccanici (MEMS). \newline
Nel 2003, Han et al. hanno usato MEMS basati su sistemi inerziali, come gli accelerometri, per esplorare le caratteristiche collegate agli episodi di FoG. Hanno trovato che la frequenza di risposta nei pazienti che indossavano gli accelerometri nella caviglia era intorno ai 6-8 Hz\cite{19}. Nel 2008, Moore et al. hanno proposto una metodologia per identificare FoG con un'accelerometro posizionato nella caviglia nella quale hanno descritto il Freezing Index (FI), ossia il quoziente del rapporto della densità spettrale di potenza (PSD) tra 3 ed 8 Hz, chiamata Freezing Band (FB), con la PSD tra 0.5 e 3 Hz, denominata Walking Band (WB)\cite{21}. Quando il FI supera una certa soglia (Freezing Threshold (FTH)), si considera che si sia verificato un episodio di FoG. A causa della presenza dei falsi positivi (FP) quando il paziente è a riposo, Bachlin et al. hanno introdotto il concetto di Power Index (PI), definito come la somma della WB e FB, il quale viene comparato con la Power Threshold (PTH) al fine di stabilire se c'era una quantità rilevante di movimento nel momento in cui il FI era alto, ossia oltre la soglia\cite{21}. PI indica la quantità di movimento, perciò situazioni nelle quali il paziente non si stesse movendo volontariamente sono state eliminate. In quest'ultima versione dell'algoritmo, quindi, un espisodio di FoG è occorso se FI>FTH e PI>PTH. Questo metodo è il più avanzato nella detenzione di FoG dato il suo scarso costo computazionale e le sue buone performance\cite{22}. \newline
L'algoritmo MBFA è stato ampiamente utilizzato nell'analisi del FoG, anche se di solito in condizioni di laboratorio e molto spesso con pochi pazienti. Jovanov et al. hanno implementato un algoritmo real time, anche se un solo volontario è stato usato per testare l'algoritmo. Inoltre, nessun risultato su sensitività e specificità è stato riportato\cite{22}. Zabaleta et al. hanno analizzato il FoG per mezzo di accelerometri a tre assi e giroscopi a due assi in differenti locazioni degli arti inferiori. La caratteristica principale ad essere stata analizzata è il FI in congiunzione con i cambiamenti della densità spettrale di potenza. Sono stati capaci di identificare correttamente l'82.7\% delgi episodi di FoG con i sensori inerziali posizionati su entrambe le caviglie, anche se in soli 2 pazienti\cite{24}.  \newline
Negli anni più recenti, Niazmand et al. (2011) hanno presentato il Mimed-Pants\cite{26}, pantaloni da jogging lavabili con 5 accelerometri integrati. Hanno usato MBFA per identificare FoG, ottenendo un 88.3\% in sensitività e 85.3\% in specificità con 6 pazienti in brevi e controllati test focalizzati nell'indurre FoG senza tenere conto dei FP. Nel 2012, Zhao et al.\cite{46} hanno sviluppato un algoritmo embedded basato sull'approccio MBFA all'interno del sistea Mimed-Pants ottenendo un 81\% in sensitività con 8 pazienti usando dei test simili ai precedenti. Più recentemente, Mazilu et al. hanno poposto un nuovo algoritmo online usando 3 accelerometri ed comparando diversi classificatori di machine learning che sfruttavano le caratteristiche del MBFA, aggiungendone di nuovo, in 10 pazienti\cite{48}. I risultati ottenuti sono stati migliori del 95\% per specificità e sensitività con differenti classificatori. Questi test, però, sono stati condotti in situazioni di controllo ed, inoltre, la metodologia di validazione sovrastimava le prestazioni delle misure poichè i classificatori erano allenati, iterativamente, con tutte le finestre del segnale disponibili da un paziente escludendone una, la quale veniva usata per ottenere le prestazioni citate. Inoltre, le sequenze di allenamento e di test erano molto simili, il che è molto diverso da normali situazioni. Quindi, ci si aspetta che le riportate specificità e sensitività calino drasticamente in situazioni non controllate. \newline
Nel 2013, Moore et al. hanno pubblicato il più recente lavoro focalizzato sul MBA. In questo, hanno confrontato differenti configurazioni applicando lo stesso algoritmo in 25 pazienti, dei quali 20 hanno avuto episodi di FoG. Diverse finestre di segnale, posizionamento dei sensori e valori per PTH e FTH sono stati valutati al fine di trovare le condizioni ottimali. I risultati migliori sono stati ottenuti con le finestre di segnale più lunghe, anche se con queste Moore et al. hanno riportato una rilevante perdita di sensitività negli episodi brevi che, paradossalmente, sono quelli più frequenti nei pazienti affetti da PD\cite{27}.
