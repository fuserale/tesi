\documentclass[a4paper, 11pt, twoside, openright, title]{book}
\usepackage[T1]{fontenc}
\usepackage[utf8]{inputenc}
\usepackage[linesnumbered,ruled,vlined]{algorithm2e}
\usepackage[italian]{babel}
\usepackage{comment}
\usepackage{lscape}
\usepackage{rotating}
\usepackage{longtable}
\usepackage{matlab-prettifier}
\usepackage{tikz}
\tikzstyle{mybox} = [draw=black, very thick, rectangle, rounded corners, inner ysep=6pt, inner xsep=6pt]

%\usepackage[suftesi]{frontespizio}

% Useful commands
\newcommand{\chapterref}[1]{Capitolo \ref{#1}} 
\newcommand{\appendixref}[1]{Appendice \ref{#1}}
\newcommand{\exampleref}[1]{Esempio \ref{#1}} 
\newcommand{\sectionref}[1]{Sezione \ref{#1}} 
\newcommand{\figureref}[1]{Figura \ref{#1}} 
\newcommand{\tableref}[1]{Tabella \ref{#1}} 
\newcommand{\grassetto}[1]{\textbf{#1}} 
\newcommand{\corsivo}[1]{\textit{#1}}

%renew
\renewcommand{\thealgocf}{}


% Colours definitions
%\usepackage[usenames,dvipsnames, table]{xcolor}
%\colorlet{myred}{red!90}
%\colorlet{mygreen}{Green!90}
%\definecolor{mygray}{rgb}{0.5,0.5,0.5}
%\definecolor{mymauve}{rgb}{0.58,0,0.82}

% Tables
\usepackage{multirow,bigdelim}

% Definitions/Theorems/Examples
\usepackage{amsthm}
\usepackage{amsfonts}
\usepackage{amsmath}
\usepackage{braket}
\theoremstyle{definition}
\newtheorem{definition}{Definition}[chapter]

% Chapter style
\usepackage{titlesec}
\newcommand{\setChapters}[1]{
\titleformat{\chapter}
  [display]{\Large\bfseries} 
  {\hfill #1 \thechapter}
  {0.05ex} {\vspace{-1.2ex}
   \rule{\textwidth}{0.4pt}
   \vspace{-2.4ex}\\}[]}

% Header style
\usepackage{fancyhdr}
\pagestyle{fancy}
\renewcommand{\headrulewidth}{0.4pt}
\renewcommand{\chaptermark}[1]%
{\markboth{\textbf{\chaptername\ \thechapter.\ #1}}{}}
\renewcommand{\sectionmark}[1]%
{\markright{\textbf{\thesection.\ #1}}}
\fancyhf{}
\fancyhead[LO]{\rightmark}
\fancyhead[RE]{\leftmark}
\fancyhead[LE,RO]{\textbf\thepage}

\fancypagestyle{plain}{%
\renewcommand{\headrulewidth}{0pt}
\fancyhf{}}

% Tikz
\usepackage{tikz}
\tikzstyle{bitvector} = [ thick,
  font=\scriptsize, align = center,
  rectangle, rounded corners]

% Listings
\usepackage{listings}


%\lstset{
%  % Margins
%  xleftmargin=0.5cm,
%  xrightmargin=0cm,
%  framexleftmargin=0.5cm,
%  framexrightmargin=0cm,
%  % Style
%  basicstyle=\ttfamily\footnotesize,  
%  keywordstyle=\bfseries,
%  commentstyle=\color{black},
%  backgroundcolor=\color{white},   % choose the background color; you must add \usepackage{color} or \usepackage{xcolor}
%  basicstyle=\footnotesize,        % the size of the fonts that are used for the code
%  breakatwhitespace=false,         % sets if automatic breaks should only happen at whitespace
%  breaklines=true,                 % sets automatic line breaking
%  captionpos=b,                    % sets the caption-position to bottom
%  %commentstyle=\color{mygreen},    % comment style
%  deletekeywords={...},            % if you want to delete keywords from the given language
%  escapeinside={\%*}{*)},          % if you want to add LaTeX within your code
%  extendedchars=true,              % lets you use non-ASCII characters; for 8-bits encodings only, does not work with UTF-8
%  %frame=single,	                   % adds a frame around the code
%  keepspaces=true,                 % keeps spaces in text, useful for keeping indentation of code (possibly needs columns=flexible)
%  keywordstyle=\color{black},       % keyword style
%  language=C++,                    % the language of the code
%  otherkeywords={*,...},           % if you want to add more keywords to the set
%  numbers=left,                    % where to put the line-numbers; possible values are (none, left, right)
%  numbersep=5pt,                   % how far the line-numbers are from the code
%  numberstyle=\tiny\color{black}, % the style that is used for the line-numbers
%  rulecolor=\color{black},         % if not set, the frame-color may be changed on line-breaks within not-black text (e.g. comments (green here))
%  showspaces=false,                % show spaces everywhere adding particular underscores; it overrides 'showstringspaces'
%  showstringspaces=false,          % underline spaces within strings only
%  showtabs=false,                  % show tabs within strings adding particular underscores
%  stepnumber=1,                    % the step between two line-numbers. If it's 1, each line will be numbered
%  stringstyle=\color{black},       % string literal style
%  tabsize=2,                   % sets default tabsize to 2 spaces
%  title=\lstname        
%}

%\newcommand{\lstcinputlisting}[1]{\lstinputlisting[style=CPP]{#1}}



% Bibliography
\usepackage[backend=bibtex, defernumbers=false, maxnames=10]{biblatex}
\usepackage{filecontents}
\bibliography{bibliography}
%\defbibfilter{papers}{type=article or type=inproceedings}
%\defbibfilter{books}{type=book}
%\defbibfilter{others}{not type=article and not type=inproceedings and not type=book}

\usepackage{fixltx2e}
\newcommand{\apsl}{A\textsubscript{P}SL}

\usepackage{enumitem}

% Referencesfa
\usepackage{hyperref}
\hypersetup{hidelinks}

% AcronymsR
\usepackage{acronym}
\usepackage{adjustbox}
\usepackage{pdfpages}

\setlength{\headheight}{14pt} 
%\setcounter{secnumdepth}{3} % seting level of numbering (default for "report" is 3). With ''-1'' you have non number also for chapters
%\setcounter{tocdepth}{3}

\begin{document}

\lstset{language=Matlab}
% ---------------------------------------------------------------
% Frontmatter 
% ---------------------------------------------------------------
\frontmatter

% First page
\includepdf{titlepage/titlepage.pdf}
\clearpage\thispagestyle{empty}
% Dedication
% Abstract
\setChapters{Capitolo}
\fancyhead[LO, RE]{\textbf{Contents}}
% !TEX root = ../main.tex

\cleardoublepage
\thispagestyle{empty}

\leavevmode \\[0.86cm]
\begin{center}
\rule{\textwidth}{.4pt} \\
\end{center}
{\LARGE\textbf{Abstract}}
\vspace{1cm}

\clearpage
\thispagestyle{empty}
% Acknowledgements
%% !TEX root = ../main.tex
\chapter{Ringraziamenti}\label{acknowledgements}

% Table of contents

\fancyhead[LO, RE]{\textbf{Contenuto}}
\tableofcontents
\listoftables
\listoffigures
% ---------------------------------------------------------------
% Mainmatter 
% ---------------------------------------------------------------
\mainmatter

% Chapters
\fancyhead[LO]{\rightmark}
\fancyhead[RE]{\leftmark}
% !TEX root = ../main.tex

% ---------------------------------------------------------------
% Introdaction
% ---------------------------------------------------------------


\chapter{Introduzione}\label{cap1:Introduzione}

% ---------------------------------------------------------------
% Introdaction
% -Motivation
% ---------------------------------------------------------------



% ---------------------------------------------------------------
% Introdaction
% -Motivation
% -Methodology
% -Thesis Contribution
% ---------------------------------------------------------------
\section{Contributo della Tesi}\label{cap1:Contributo della Tesi}
In this thesis, we have developed a methodology 

% ---------------------------------------------------------------
% Introdaction
% -Motivation
% -Methodology
% -Thesis Contribution
% -Outline
% ---------------------------------------------------------------

\section{Struttura della Tesi}\label{cap1:Struttura della Tesi}
Il resto della tesi è organizzata nel modo seguente:
\begin{itemize}
	\item \chapterref{chap2:related}
\end{itemize}
 
% !TEX root = ../main.tex

% ---------------------------------------------------------------
% RELATED WORKS
% ---------------------------------------------------------------

\chapter{Letteratura}\label{chap2:related}
Il problema del FOG è stato analizzato tramite una grande varietà di sistemi e sensori. Alcuni di questi, però, non sono utilizzabili durante la vita quotidiana dei pazienti poiché posso essere disponibili solo in ambienti di laboratorio. Esempi di questi sistemi sono le piattaforme di pressione\cite{38}, le quali sono non portatili, l'elettromiografia (EMG)\cite{25}, l'elettroencefalogramma (EEG)\cite{42} o la conduttanza della pelle\cite{43}, il quale comporta il piazzamento di elettrodi sulla pelle in aggiunta ad sistema di rilevamento per raccogliere i dati.
Altri sistemi invasivi sono i goniometri a ginocchio\cite{23} o sistemi che fanno uso di camere e video, i quali hanno una bassa tolleranza del paziente in un'ambiente che non sia di laboratorio\cite{23,39,44}. Quindi, dato che il monitoraggio del PD dovrebbe essere deambulatorio e  durare diverse ore al fine di ricavare utili informazioni cliniche\cite{34,45}, la maggior parte dei lavori si è basata su sistemi non invasivi come i dispositivi indossabili basati su circuiti microelettromeccanici (MEMS). \newline
Nel 2003, Han et al. hanno usato MEMS basati su sistemi inerziali, come gli accelerometri, per esplorare le caratteristiche collegate agli episodi di FoG. Hanno trovato che la frequenza di risposta nei pazienti che indossavano gli accelerometri nella caviglia era intorno ai 6-8 Hz\cite{19}. Nel 2008, Moore et al. hanno proposto una metodologia per identificare FoG con un'accelerometro posizionato nella caviglia nella quale hanno descritto il Freezing Index (FI), ossia il quoziente del rapporto della densità spettrale di potenza (PSD) tra 3 ed 8 Hz, chiamata Freezing Band (FB), con la PSD tra 0.5 e 3 Hz, denominata Walking Band (WB)\cite{21}. Quando il FI supera una certa soglia (Freezing Threshold (FTH)), si considera che si sia verificato un episodio di FoG. A causa della presenza dei falsi positivi (FP) quando il paziente è a riposo, Bachlin et al. hanno introdotto il concetto di Power Index (PI), definito come la somma della WB e FB, il quale viene comparato con la Power Threshold (PTH) al fine di stabilire se c'era una quantità rilevante di movimento nel momento in cui il FI era alto, ossia oltre la soglia\cite{21}. PI indica la quantità di movimento, perciò situazioni nelle quali il paziente non si stesse muovendo volontariamente sono state eliminate. In quest'ultima versione dell'algoritmo, quindi, un espisodio di FoG è occorso se FI>FTH e PI>PTH. Questo metodo è il più avanzato nella detenzione di FoG dato il suo scarso costo computazionale e le sue buone performance\cite{22}. \newline
L'algoritmo MBFA è stato ampiamente utilizzato nell'analisi del FoG, anche se di solito in condizioni di laboratorio e molto spesso con pochi pazienti. Jovanov et al. hanno implementato un algoritmo real time, anche se un solo volontario è stato usato per testare l'algoritmo. Inoltre, nessun risultato su sensitività e specificità è stato riportato\cite{22}. Zabaleta et al. hanno analizzato il FoG per mezzo di accelerometri a tre assi e giroscopi a due assi in differenti locazioni degli arti inferiori. La caratteristica principale ad essere stata analizzata è il FI in congiunzione con i cambiamenti della densità spettrale di potenza. Sono stati capaci di identificare correttamente l'82.7\% delgi episodi di FoG con i sensori inerziali posizionati su entrambe le caviglie, anche se in soli 2 pazienti\cite{24}.  \newline
Più recentemente, Niazmand et al. (2011) hanno presentato il Mimed-Pants\cite{26}, pantaloni da jogging lavabili con 5 accelerometri integrati. Hanno usato MBFA per identificare FoG, ottenendo un 88.3\% in sensitività e 85.3\% in specificità con 6 pazienti in brevi e controllati test focalizzati nell'indurre FoG senza tenere conto dei FP. Nel 2012, Zhao et al.\cite{46} hanno sviluppato un algoritmo embedded basato sull'approccio MBFA all'interno del sistema Mimed-Pants ottenendo un 81\% in sensitività con 8 pazienti usando dei test simili ai precedenti. Più recentemente, Mazilu et al. hanno proposto un nuovo algoritmo online usando 3 accelerometri ed comparando diversi classificatori di machine learning che sfruttavano le caratteristiche del MBFA, aggiungendone di nuovi, in 10 pazienti\cite{48}. I risultati ottenuti sono stati migliori rispetto ai precedenti, con un 95\% per specificità e sensitività con differenti classificatori. Questi test, però, sono stati condotti in situazioni di controllo ed, inoltre, la metodologia di validazione sovrastimava le prestazioni delle misure poiché i classificatori erano allenati, iterativamente, con tutte le finestre del segnale disponibili da un paziente escludendone una, la quale veniva usata per ottenere le prestazioni citate. Inoltre, le sequenze di allenamento e di test erano molto simili, il che è molto diverso da normali situazioni. Quindi, ci si aspetta che le riportate specificità e sensitività calino drasticamente in situazioni non controllate. Sempre Mazilu et al, hanno ipotizzato l'esistenza di una terza classe, da loro chiamata preFoG, che si posizionerebbe prima delle occorrenze di FoG e racchiude tutti i movimenti che portano al Freezing. Tentando un approccio di cross-validation con questa nuova classe, é stata raggiunta una misura del 70\% per quanto rigurda la F1-measure.\cite{12} \newline
Nel 2013, Moore et al. hanno pubblicato il più recente lavoro focalizzato sul MBA. In questo, hanno confrontato differenti configurazioni applicando lo stesso algoritmo in 25 pazienti, dei quali 20 hanno avuto episodi di FoG. Diverse finestre di segnale, posizionamento dei sensori e valori per PTH e FTH sono stati valutati al fine di trovare le condizioni ottimali. I risultati migliori sono stati ottenuti con le finestre di segnale più lunghe, anche se con queste Moore et al. hanno riportato una rilevante perdita di sensitività negli episodi brevi che, paradossalmente, sono quelli più frequenti nei pazienti affetti da PD\cite{27}. In un test più complesso eseguito precedentemente\cite{20} usando fino a 7 sensori ed un protocollo di test più lungo, sono stati ottenuti una sensitività e specificità sopra al 70\%, anche se, in certe configurazioni (finestra di segnale pari a 7.5s e il sistema installato nella zona lombare), sia per per la sensitività che la specificità hanno raggiunto valori oltre l'80\%. In un approccio differente, Tripoli et al. hanno testato diverse configurazioni e locazioni dei sensori al fine di trovare la migliore configurazione\cite{52}. Il lavoro è stato svolto con 5 pazienti ed in condizioni controllate, usando uno specifico protocollo progettato per stimolare il FoG e senza test di FP. In tale lavoro, hanno integrato 2 giroscopi oltre a 6 accelerometri posizionati in posizioni differenti del corpo. Con tutti i sensori indossati, è stata ottenuta un'accuratezza del 96.11\%, una specificità del 98.74\% e, eseguendo i test su tutti i pazienti tranne uno, una sensitività dell'81.94\%. D'altra parte, con una IMU singola nella zona lombare hanno riportato una sensitività del 75\% ed una specificità del 95\%, anche se l'algoritmo non è stato confrontato con nessun altro metodo usato sotto le stesse condizioni. \newline
Mazilu et al.\cite{50} hanno investigato un approccio di apprendimento non supervisionato per costruire un input ottimale per un classificatore ad albero di decisione con il dataset del progetto DAPHNET (10 pazienti PD). Il loro approccio è stato comparato ad un analogo basato su MBFA nel quale il FI e l'energia della banda spettrale tra 0.5 Hz e 8 Hz sono state valutate. L'allenamento ed i test erano dipendenti dall'utilizzatore e sotto condizioni controllate. I risultati superano l'approccio MBFA similare dell'8.1\% in termini di punteggio dell'F1. Un altro approccio è stato presentato da Rodriguez et al., i quali hanno proposto un metodo per contestualizzare gli episodi di FoG tramite un algoritmo di riconoscimento dell'attività, il quale rifiutava i FP quando il paziente era seduto o eseguiva attività quali disegnare o digitare in un laptop. La specificità è stata aumentata in media del 5\%, arrivando anche ad un 11.9\% in certi casi\cite{50}. Il metodo che aggiungeva la contestualizzazione, però, non ha contribuito a migliorare la sensitività. Altri studi hanno la variabilità della camminata tra un episodio di FoG e condizioni normal. Anche se i risultati sono interessanti, hanno fallito nell'includere i falsi positivi ed un'affidabile classificazione non è stata eseguita\cite{53,54}. Un paper recente di Zach et al. presenta una nuova metodologia per suscitare FoG in condizioni di laboratorio controllate, le quali sono state valutate con l'algoritmo MBFA ottenendo una sensitività del 75\% ed una specificità del 76\%\cite{31}. \newline
Infine, Alrichs et al., all'interno del progetto REMPARK\cite{55}, usano una Support Vector Machines (SVM) per rilevare episodi di FoG in 8 pazienti con PD in ambiente casalingo. Il metodo include test in differenti test motori usando un singolo accelerometro nella zona lombare, raggiungendo un'accuratezza del 90\%. La specificità, però, è stata calcolata solo non pazienti non FoG, il che può portare a predizioni non affidabili in quanto il modello non è stato testato con pazienti PD con FoG, i quali hanno movimenti molto diversi dai pazienti che non soffrono di FoG. Inoltre, la valutazione è stata eseguita su finestre di un minuto, tempo che è considerato troppo lungo per un'implementazione online\cite{28}. Sempre all'interno del progetto REMPARK, Rodriguez et al. hanno presentato un lavoro che utilizza un algoritmo per rilevare FoG tramite un approccio di machine learning basato su SVM ed un singolo accelerometro a 3 assi indossato nella zona lombare\cite{HD}.Il metodo è stato valutato su 21 pazienti affetti da PD in ambienti casalinghi sotto due condizioni: un modello generico testato su tutti i pazienti tranne uno ed un secondo modello personalizzato sull'utente che usa parte del dataset del paziente stesso. I risultati mostrano un significativo vantaggio del modello personalizzato rispetto a quello generico, portando ad un miglioramento in media, sia della sensitività che della specificità, del 7.2\%. Inoltre, l'approccio adottato è stato comparato con i metodi più utilizzati per la detenzione del FoG basati sull'algoritmo MBFA. I risultati del metodo generico mostrano un miglioramento in media dell'11.2\% rispetto a metodi MBFA generici, mentre quello personalizzato porta ad un miglioramento del 10\% rispetto ad altri metodi specifici sul paziente.
% !TEX root = ../main.tex

% ---------------------------------------------------------------
% Background
% ---------------------------------------------------------------

\chapter{Background}\label{chap3:background}

% !TEX root = ../main.tex
% ---------------------------------------------------------------
% GOALS
% ---------------------------------------------------------------



\chapter[Obiettivi]{Obiettivi}\label{chap4:Goals}
Allo stato dell'arte, riassunto nella tabella \ref{letteratura}, sono presenti molti lavori di classificazione del Freezing, usando principalmente 2 classi, ossia noFOG, che corrisponde ad attività definite normali del paziente, e FOG, ossia un blocco motorio. Uno studio ha cercato di introdurre una nuova classe, intermedia tra le due, che é stata chiamata preFOG. Questa rappresenta una fase transitoria da uno stato di noFOG ad un'occorrenza di FOG. Identificare tale classe permetterebbe di prevedere i Freezing del paziente e quindi permettergli di non bloccarsi attraverso uno stimolo, uditorio o visivo.\\
Gli studi condotti finora, inoltre, utilizzano dei dataset composti da dati ricavati tramite accelerometri ed etichettati manualmente da dottori. Non é stato ancora presentato un approccio che tenti di sostituire il lavoro di etichettatura dei dati del medico.\\
La tesi proposta si prefigge lo scopo di presentare un approccio non supervisionato per l'etichettatura dei dati provenienti da accelerometri senza l'ausilio del medico ed usare un approccio di classificazione per identificare le occorrenze di preFOG al fine di prevedere i Freezing.\\
\begin{tikzpicture}
\node [mybox] (box){%
	\begin{minipage}{.96\textwidth}
		Gli obiettivi della tesi quindi sono:
			\begin{itemize}
				\item Verificare l'esistenza della classe preFOG;
				\item Usare un approccio non supervisionato per l'etichettatura dei dati e fornire uno studio di divisione degli intervalli temporali dei dati;
				\item Classificare i dati per identificare le occorrenze di preFOG e fornire uno stimolo sensoriale per evitare FOG.
			\end{itemize}
	\end{minipage}
};
\end{tikzpicture}%
%Il lavoro condotto, partendo da dati raccolti dagli assi x,y,z di più accelerometri, verifica la presenza di una nuova tipologia di classe, che chiameremo preFOG, dei dati e conduce uno studio algoritmico non supervisionato su feature ed intervalli temporali sui dati per etichettare tali dati, sostituendo in tale modo il lavoro del medico. Inoltre, introduce un metodo di classificazione dei dati stessi, al fine di identificare tale classe per evitare le occorrenze del Freezing.\\
%\begin{tikzpicture}
%\node [mybox] (box){%
%	\begin{minipage}{.96\textwidth}
%	Quello che, dunque, si vuole studiare è innanzitutto l'esistenza di una nuova classe chiamata preFoG e che sia divisa da quelle di FoG e noFoG. Una volta verificato questo, si vuole sviluppare un procedimento di riconoscimento non supervisionato di etichette delle varie classi attraverso algoritmi di clustering, sostituendo cosí il medico nella prima fase di test. Sfruttando tale intervallo, inoltre, si vuole tentare un approccio di classificazione per identificare le occorrenze di preFOG.
%	\end{minipage}
%};
%\end{tikzpicture}%
\\ La figura \ref{FlussoTesiGenerale} rappresenta il flusso degli obiettivi della tesi. La fase 1, da dati di accelerometri, vuole verificare, tramite uno studio dei dati, la distinzione della classe preFOG dalle altre. 
La fase 2, usando tale informazione, applica un approccio supervisionato al fine di etichettare i dati senza l'ausilio di un medico e conduce uno studio sugli intervalli temporali, ossia il modo migliore per dividere i dati di ingresso degli accelerometri. La fase 3, usando le informazioni sulle finestre di suddivisione temporale dei dati della fase precedente, si pone l'obiettivo di usare un classificatore per generare uno stimolo sensoriale al fine di evitare FOG.
\begin{figure}[]
	\centering
	\includegraphics[scale=0.46]{images/FlussoTesiGenerale.png}
	\caption{Rappresentazione del flusso generale della tesi}
	\label{FlussoTesiGenerale}
\end{figure}

% !TEX root = ../main.tex

% ---------------------------------------------------------------
% Automatic generation of a self-adaptive TLM model
% ---------------------------------------------------------------

\chapter[Apprendimento non supervisionato]{Apprendimento non supervisionato per l'identificazione di contensti di FoG}\label{chap5:Automatic}
I lavori che sono stati svolti e presentati nel capitolo \ref{chap2:related} sul Freezing Of Gait sono basati sull'identificazione di 2 classi, ossia quando il paziente e' in una normale attivita' quotidiana, che chiameremo NoFoG, oppure quando il paziente soffre di un episodio di FoG. Il nostro lavoro, invece, vuole basarsi sull'identificazione anche di una nuova classe rispetto alle 2 precedenti, che chiamiamo preFoG. Questa dovrebbe rappresentare la fase di transizione da uno stato di NoFoG ad uno di FoG, per cui identificare tale classe permetterebbe di anticipare l'occorrenza di un FoG e, quindi, dare in anticipo uno stimolo al paziente per evitare il verificarsi del FoG. La durata di ogni preFoG e' stata fissata a 2 secondi in quanto, secondo noi, questo intervallo di tempo e' sufficiente per raccogliere abbastanza informazioni al fine di identificare le occorrenze di preFoG e quindi quelle di FoG in anticipo rispetto al suo verificarsi. Il flusso della tesi viene rappresentato in figura \ref{FlussoTesi}.\\
\begin{figure}[]
	\centering
	\includegraphics[scale=0.45]{images/FlussoTesi.png}
	\caption{Rappresentazione del flusso della tesi}
	\label{FlussoTesi}
\end{figure}
Lo sviluppo della procedura e' stato svolto usando software Matlab.
\section{Dataset}
L'approccio che andiamo a proporre è stato testato sul dataset DAPHNET\footnote{www.wearable.ethz.ch/resources/Dataset}, il quale contiene dati collezionati da 10 pazienti parkinsoniani, dei quali 8 presentano contesti di FoG, mentre 2 di loro non ne presentano. I dati sono stati registrati usando 3 accelerometri 3D attaccati alla caviglia, al ginocchio e nella zona lombare del paziente, usando una frequenza di campionamento di 64 Hz, ossia vengono raccolti 64 campioni ogni secondo.\\
I soggetti hanno completato sessioni da 20-30 minuti ciascuno, consistenti di 3 fasi di camminata:
\begin{enumerate}
	\item Camminata avanti ed indietro lungo una linea retta, con delle rotazioni di 180 gradi;
	\item Camminata casuale con una serie di fermate volontarie e rotazioni di 360 gradi;
	\item Camminata che simula attività di vita quotidiana, tra le quali entrare in stanze ed uscirne, camminare nella cucina, prendersi un bicchiere d'acqua e tornare al punto di partenza.
\end{enumerate}
Le prestazioni motorie variano molto tra i pazienti. Mentre alcuni soggetti hanno mantenuto una camminata regolare durante gli episodi di non FoG, altri hanno camminato molto lentamente ed in modo instabile. L'intero dataset contiene in totale 237 episodi di FoG; la durata di ognuno di essi è tra i 0.5s ed i 40.5s. Il 50\% degli episodi di FoG è durato meno di 5.4s ed il 93.2\% è più corto di 20s. Gli episodi di FoG sono stati identificati da fisioterapisti usando registrazioni video sincronizzate. L'inizio di un episodio di FoG è stato definito come il punto dove la sequenza normale di camminata è stata interrotta, mentre la fine del FoG è stata definita come il momento in cui tale sequenza riprende.

\section{Dimostrazione dell'esistenza effettiva delle 3 classi}
Il lavoro svolto in questa tesi, come scritto precedentemente, si differenzia dagli altri gia' citati in \ref{chap2:related} in quanto si vogliono usare 3 classi invece delle 2 fornite dal dataset. Il primo passo dunque e' quello di dimostrare che effettivamente le 3 classi sono distinte tra loro e che quindi sviluppare un approccio basandosi su tali classi e' possibile.\\
Per fare questo, si e' usato un approccio di Discriminazione Lineare. Un discriminante e' una funzione che prende in ingresso un vettore x e lo assegna ad una tra le K classi fornite. Il metodo di discriminazione che si vuole usare e' quello di Fisher, il quale viene spiegato nel Capitolo \ref{chap3:background}.\\
\subsection{Divisione dei dati}
Per ogni paziente, prendiamo in input la matrice di dati grezzi degli accelerometri fornita dal dataset, la quale contiene anche un campo relativo al tempo di ogni campione ed un'etichetta, che mi indica lo stato in cui si trova il mio paziente. Dividiamo i dati degli accelerometri da quelli relativi al tempo ed allo stato e calcoliamo delle feature su tali dati. Queste vengono calcolate vettorizzando finestre di dati grezzi, dove ognuna di esse dura 2 secondi, in accordo con quanto posto per la durata del pre-FoG. Ogni nuova finestra non comincia dalla fine della precedente, ma presenta una certa sovrapposizione. Per tenere traccia dell'etichetta reale a cui ogni finestra temporale appartiene, abbiamo usato il vettore relativo all'etichetta fornita dal dataset e, per ogni finestra temporale, abbiamo tenuto quella che si presenta piu' volte all'interno di essa. Il codice implementativo di quanto appena descritto e' fornito in \ref{vettorizzazione}.
\begin{lstlisting}[style=Matlab-editor,frame=single, caption=Vettorizzazione dei dati degli accelerometri, label=vettorizzazione]
for i=1:size_windows_sample-size_overlap_samples:m - size_windows_sample
B = A(i:i+size_windows_sample-1,:);
% vettorizzo ogni finestra
B=B(:);
F(number_sample,:)=B';


%salvo la classe di ogni finestra
class(number_sample)=mode(FREEZE(i:i+size_windows_sample-1,:));

%go to next sample
number_sample = number_sample + 1;

end
\end{lstlisting}
\subsection{Algoritmo di discriminazione}
Una volta vettorizzato l'intero dataset, si procede ad applicare l'algoritmo di discriminazione, il quale per prima cosa divide le feature appena calcolate tramite le classi di appartenenza. Calcola quindi la media per ogni classe e determina la dimensione di ognuna di esse. A questo punto, determina la within class covariance e la between class covariance, ossia quanta similarita' esiste tra gli elementi della stessa classe e quanta dissimilarita' e' presente tra classi differenti. Una volta trovate tali misure, prende gli autovalori ed autovettori della soluzione del problema dell'autovalore generalizzato e teniamo solo quelli piu' rappresentativi, che nel caso dell'approccio di Fisher sono pari al numero di classi presenti meno 1. Questi autovalori vengono poi moltiplicati per le feature calcolate precedentemente per ottenere una riduzione della dimensionalità' ed avere 2 feature rappresentative. Il codice della discriminazione viene fornito in \ref{code_LDA}.
\begin{lstlisting}[style=Matlab-editor,frame=single, caption=LDA, label=code_LDA]
A=F';

%quindi ho A 1152x1449 double
%1152 finestre
%ogni finestra ha 1449 features
[d,N] = size(A);

K =  max(class); % numero classi in gioco;

% 1. Divido le feature tramite le classi Ck
for k = 1:K
a = find (class == k);
Ck{k} = A(:, a);
end

% 2. Calcolo la media per ogni classe per ogni finestra
for k = 1:K
mk{k} = mean(Ck{k},2);
end
% 3. Determino la grandezza di ogni classe
for k = 1:K
[d, Nk(k)] = size(Ck{k});
end
% 4. determino le within class covariance
for k = 1:K
S{k} = 0;
for i = 1:Nk(k)
S{k} = S{k} + (Ck{k}(:,i)-mk{k})*(Ck{k}(:,i)-mk{k})';
end
S{k} = S{k}./Nk(k);
end
Swx = 0;
for k = 1:K
Swx = Swx + S{k};
end

% 5. determino la between class covariance
% 5.1 determino la media totale
m = mean(A,2);
Sbx = 0;
for k=1:K
Sbx = Sbx + Nk(k)*((mk{k} - m)*(mk{k} - m)');
end
Sbx = Sbx/K;

MA = inv(Swx)*Sbx;

% eigenvalues/eigenvectors
[V,D] = eig(MA);

% 5: transform matrix
if (k > 1)
W = V(:,1:K-1);
end
if (k == 1)
W = V(:,1:1);
end

% 6: transformation
Y = W'*A;
\end{lstlisting}
\subsection{Risultati}
Una volta calcolate le feature discriminanti, per visualizzare se effettivamente c'e' una distinzione tra le classi, facciamo un grafico con queste due grandezze ed ogni punto viene colorato in base alla reale classe di appartenenza. La figura \ref{LDAS01} mostra tale grafico. Si puo' notare immediatamente come le due feature utilizzate riescono a discriminare effettivamente tra le 3 classi e quindi un approccio basandosi su tali componenti e' giustificato.
\begin{figure}[]
	\centering
	\includegraphics[scale=0.2]{images/LDAS01.png}
	\caption{Grafico delle 3 classi in base alla discriminazione data dall'approccio di Fisher per il paziente 1}
	\label{LDAS01}
\end{figure}
\section{Determinazione del miglior intervallo di finestra}
In letteratura, la finestra temporale su cui vengono calcolate le feature e' sempre stata presa a priori, basandosi su delle ipotesi. Non e' mai stato condotto quindi uno studio su quale possa essere il miglior intervallo di divisione dei dati per calcolare le feature. Quello che ci prefissiamo in questa sezione e' dunque testare diverse combinazioni di durata delle finestre con sovrapposizione al fine di trovare quella che meglio divide i nostri dati. Per fare questo, abbiamo utilizzato algoritmi di clustering in quanto non volevamo che il risultato fosse influenzato dall'appartenenza o meno di un dato ad una determinata classe, come nel caso dell'LDA o di algoritmi di classificazione. L'approccio che è stato preso in considerazione è basato sul calcolo di feature (o proprietà) derivanti dalla matematica statistica. Il flusso di tale approccio è:
\begin{enumerate}
	\item Pre-processamento dei dati degli accelerometri, al fine di eliminare il rumore presente ed identificare eventuali punti di outline, ossia campioni che non presentano affinità col resto dei dati poiché dovuti a movimenti non consoni;
	\item Finestramento dei dati in base ad intervalli variabili al fine di calcolare le feature, dove ogni intervallo presenta una certa sovrapposizione con l'intervallo precedente;
	\item Calcolo effettivo delle feature statistiche;
	\item Applicazione degli algoritmi di clustering;
	\item Calcolo di metriche che indicano quanto bene gli algoritmi di clustering hanno diviso i dati in relazione alle label fornite dal dataset.
\end{enumerate}
\begin{figure}[h!]
	\centering
	\includegraphics[scale=0.6]{images/flusso_feature.png}
	\caption{Schema generale di calcolo delle feature statistiche}
	\label{Flusso Feature}
\end{figure}
\subsection{Pre-processamento dei dati}
Gli accelerometri sono dispositivi che misurano le vibrazioni o l'accelerazione del movimento di una struttura. La forza generata dalle vibrazioni o una variazione del movimento (accelerazione) fa in modo che la massa "comprima" il materiale piezoelettrico, che genera una carica elettrica proporzionale alla forza esercitata su di esso. Dato che la carica è proporzionale alla forza e che la massa è una costante, la carica è proporzionale anche all'accelerazione. Come tutti i dispositivi che misurano delle grandezze presentano dell'incertezza strumentale che può portare ad avere rumore, ossia un segnale non desiderato, di origine naturale o artificiale, che si sovrappone all'informazione degli accelerometri stessi. QUesto rumore porta ad avere dei punti definiti outlier, ossia campioni che non presentano affinità col resto dei dati poiché dovuti a movimenti non consoni.\\
Al fine di rimuovere tali punti, che altererebbero in modo negativo il calcolo delle nostre feature, si rende necessario rimuoverli dal nostro dataset. Per identificarli, è stato implementato un filtro passa-alto che eliminano tutte le frequenze inferiori a 0.5Hz, le quali non appartengono al normale movimento umano ma indicano appunto la presenza di rumore, come evidenziato in \cite{21}. L'implementazione del filtro viene fornita in \ref{hpfilter}.
\begin{lstlisting}[style=Matlab-editor,frame=single, caption=hpfilter, label=hpfilter]  % Start your code-block
function Hd = hpfilter
% All frequency values are in Hz.
Fs = 64;  % Sampling Frequency

Fstop = 0.4;    % Stopband Frequency
Fpass = 0.8;       % Passband Frequency
Astop = 60;      % Stopband Attenuation (dB)
Apass = 1;       % Passband Ripple (dB)
match = 'passband';  % Band to match exactly

% Construct an FDESIGN object and call its ELLIP method.
h  = fdesign.highpass(Fstop, Fpass, Astop, Apass, Fs);
Hd = design(h, 'cheby2', 'MatchExactly', match);
\end{lstlisting}
\subsection{Definizione degli intervalli}
Una volta filtrati i dati, il passo successivo dell'algoritmo consiste nel dividerli in intervalli temporali, al fine di poter computare le caratteristiche degli stessi. Un intervallo contiene un determinato movimento, che però verrebbe "interrotto" con la fine della finestra stessa e quindi potrei perdere informazioni su tale movimento. Per evitare il più possibile tale perdita, è usato usato un approccio che sfrutta le sovrapposizioni tra intervalli, per cui ogni nuova finestra non inizia subito dopo la fine della precedente, ma all'interno di essa. La durata degli intervalli che sono stati presi in considerazione sono:
\begin{itemize}
	\item  Da 1 secondi a 5 secondi per la finestra temporale, con un incremento di 0.5 secondi;
	\item Da 0.5 a 4.5 secondi per l'intervallo di overlap, con un incremento sempre di 0.5 secondi.
\end{itemize}
La condizione fondamentale per usare le sovrapposizioni è che l'intervallo delle stesse non sia mai maggiore della durata delle finestre temporali, per cui è stata posta la condizione che la finestra temporale sia sempre almeno 0.5 secondi più lunga dell'intervallo di sovrapposizione. Per capire da dove parte la nuova finestra, quindi, prendiamo la posizione a cui siamo arrivati e togliamo la durata della sovrapposizione. L'implementazione viene fornita in \ref{intervals}.
\begin{lstlisting}[style=Matlab-editor,frame=single, caption=Definizione degli intervalli, label=intervals]  % Start your code-block
...
for k = 5:5:45

Y = k/10;

for i = (Y+0.5):0.5:5
%dimensione della finestra in secondi
size_windows_sec = i;
%dimensione della finestra in campioni
size_windows_sample = Fs * i;

%dimensione dell'overlap in secondi
size_overlap_sec = Y;
%dimensione dell'overlap in campioni
size_overlap_samples = Fs * Y;

number_sample = 1;

%for each sample window, compute the features
for i=1:size_windows_sample-size_overlap_samples:m - size_windows_sample
B = A(i:i+size_windows_sample-1,:);
...
\end{lstlisting}
\subsection{Calcolo delle Feature}
 Per ogni possibile combinazione di finestra temporale e sovrapposizione, a questo punto vengono calcolate le feature per ogni intervallo. Le feature prese in considerazione nel nostro studio sono descritte in tabella \ref{TAB:Feature}.
\begin{table}[h!]
	\begin{tabular}{ |p{0.05\textwidth} | p{0.25\textwidth} | p{0.6\textwidth} | }
		\multicolumn{1}{|p{0.05\textwidth} |}{\textbf{N}} &  
		\multicolumn{1}{p{0.25\textwidth} |}{\textbf{Feature}} &
		\multicolumn{1}{p{0.6\textwidth} |}{\textbf{Descrizione}}\\
		\hline
		\hline
		1 & Minimo & Valore minimo del segnale\\
		2 & Massimo & Valore massimo del segnale \\
		3 & Mediana & Valore mediano del segnale \\
		4 & Media & Valore medio del segnale \\
		5 & Media Armonica & Media armonica del segnale \\
		6 & Errore Quadratico Medio & Valore Quadratico medio del segnale \\
		7 & Media Geometrica & Media geometrica del segnale \\
		8 & Varianza & Radice della deviazione standard \\
		9 & Deviazione Standard & Deviazione media del segnale rispetto alla media \\
		10 & Curtosi & Allontanamento dalla normalità distributiva del segnale \\
		11 & Simmetria & Grado di asimmetria della distribuzione del segnale \\
		12 & Moda & Il numero che appare più volte nel segnale \\
		13 & Media Tagliata & Media tagliatadel segnale nella finestra \\
		14 & Entropia & Misura della di distruzione delle componenti in frequenza \\
		15 & Range & Differenza tra il valore minimo e massimo del segnale \\
		16 & Magnitudine & Somma della norma euclidea di tre assi normalizzato sulla lunghezza del segnale \\
		17 & Area Magnitudine & Accelerazione della magnitudine di tre assi normalizzato sulla lunghezza del segnale \\
		18 & Autovalori delle direzioni dominanti & Autovalori della matrice di covarianza di tre assi \\
		19 & Accelerazione media dell'energia & Valore medio dell'energia sui 3 assi \\
	\end{tabular}
\caption{Descrizione delle Feature Statistiche}
\label{TAB:Feature}
\end{table}\\
Nel dataset, ogni campione è etichettato. Poiché le nostre feature sono composte da molti campioni messi assieme, si rende necessario trovare un metodo per unire, sbagliando il meno possibile, tutte le etichette della finestra che prendiamo in considerazione. Per fare ciò, si è deciso di utilizzare la funzione di moda, ossia il numero che si ripete più spesso all'intervallo dell'intervallo considerato. L'etichetta risultante viene inserita nella tabella e verrà utilizzata come criterio di confronto, al fine di valutare le prestazioni degli algoritmi di clustering. L'implementazione del codice è fornito in \ref{extract_feature}
\begin{lstlisting}[style=Matlab-editor,frame=single, caption=Calcolo delle Feature, label=extract_feature]  % Start your code-block
...
%time sample
F(number_sample, 1) = TIME(i,:);
%min --> minimum value for each accelerometer
F(number_sample, 2:10) = min(B);
%max --> maximum value for each accelerometer
F(number_sample, 11:19) = max(B);
%median --> median signal value
F(number_sample, 20:28) = median(B);
%mean --> average value
F(number_sample, 29:37) = mean(B);
%ArmMean --> harmonic average of the signal
F(number_sample, 38:46) = harmmean(B);
%root mean square --> quadratic mean value of the signal
F(number_sample, 47:55) = rms(B);
%variance --> square of the standard deviation
F(number_sample, 56:64) = var(B);
%standard deviation --> mean deviation of the signal compared to the
%average
F(number_sample, 65:73) = std(B);
%kurtosis --> degree of peakedness of the sensor signal distribution
%(allontanamento dalla normalita distributiva)
F(number_sample, 74:82) = kurtosis(B);
%skewdness --> degree of asymmetry of the sensor signal distribution
F(number_sample, 83:91) = skewness(B);
%mode --> number that appears most often in the signal
F(number_sample, 92:100) = mode(B);
%trim mean --> trimmed mean of the signal in the window
F(number_sample, 101:109) = trimmean(B,10);
%range --> difference between the largest and the smallest values of
%the signal
F(number_sample, 110:118) = range(B);
%signal magnitude vector --> sum of the euclidean norm over the three
%axis over the entire window normalized by the windows lenght
F(number_sample, 119) = svmn(B(:,1:3), length(B));
F(number_sample, 120) = svmn(B(:,4:6), length(B));
F(number_sample, 121) = svmn(B(:,7:9), length(B));
%normalized signal magnitude area --> acceleration magnitude summed
%over three axes normalized by the windows length
F(number_sample, 122) = sman(B(:,1:3), length(B));
F(number_sample, 123) = sman(B(:,4:6), length(B));
F(number_sample, 124) = sman(B(:,7:9), length(B));
%eigenvalues of dominant directions --> eigenvalues of the
%covariance matrix of the acceleration data along x, y and z axis
F(number_sample,125) = eigs(cov(B(:,1:3)),1);
F(number_sample,126) = eigs(cov(B(:,4:6)),1);
F(number_sample,127) = eigs(cov(B(:,7:9)),1);
%averaged acceleration energy --> mean value of the energy over
%three acceleration axes
F(number_sample,128) = energyn(B(:,1:3),length(B));
F(number_sample,129) = energyn(B(:,4:6),length(B));
F(number_sample,130) = energyn(B(:,7:9),length(B));
%is freezing?
F(number_sample,131) = mode(FREEZE(i:i+size_windows_sample-1,:));

%go to next sample
number_sample = number_sample + 1;
...
\end{lstlisting}
\subsection{Clustering}
Avendo il nuovo dataset composto dalle caratteristiche dei dati originali scomposti in intervalli, si possono applicare gli algoritmi di clustering. Quelli che sono stati presi in considerazione nella nostra analisi sono: K-means, Fuzzy C-Means e Neural Network. Il K-means viene testato in 4 varianti diverse, in base alle seguenti metriche di distanza: cityblock, correlation, cosine, euclidean.\\
Ogni algoritmo di clustering viene applicato in tutte le possibili combinazioni di intervalli ed ognuno di essi restituisce un vettore di numeri, che corrispondono alle etichette che assegnano ad ogni vettore di feature. Questi vengono salvati e verranno poi utilizzati per calcolare le prestazioni degli algoritmi stessi in confronto alla classificazione originale del dottore. L'implementazione del codice è fornita in \ref{clustering}
\begin{lstlisting}[style=Matlab-editor,frame=single, caption=Uso degli algoritmi di clustering, label=clustering]  % Start your code-block
...
            %% k-means %%%

% choose of parameter
means1 = 'sqeuclidean';
means2 = 'correlation';
means3 = 'cityblock';
means4 = 'cosine';
for q=1:4
if q == 1
dist_k = means1;
end
if q == 2
dist_k = means2;
end
if q == 3
dist_k = means3;
end
if q == 4
dist_k = means4;
end
options_km = statset('UseParallel', false);
maxiter = 100000;
% cluster
kidx = kmeans(bonds, numClust, 'distance', dist_k, 'options', options_km, 'MaxIter', maxiter);

P = array2table([A(:,n) kidx]);
writetable(P, [datadir 'versus_kmeans_' dist_k '_' fileruns(r).name] );
display([datadir 'versus_kmeans_' dist_k '_' fileruns(r).name]);
end

%%% neural networks - Self organizing Maps %%%

% Create a Self-Organizing Map
dimension1 = 3;
dimension2 = 1;
net = selforgmap([dimension1 dimension2]);

% Train the network
net.trainParam.showWindow = 0;
[net,tr] = train(net,bonds');

% Test the network
nidx = net(bonds');
nidx = vec2ind(nidx)';

P = array2table([A(:,n) nidx]);
writetable(P, [datadir 'versus_net_' fileruns(r).name] );
display([datadir 'versus_net_' fileruns(r).name]);


%     %%% FUZZY C-MEANS %%%
options(1) = 2;
options(2) = 10000;
options(3) = 1e-5;
options(4) = 0;
% Hide iteration information by passing appropriate options to FCM
[centres,U] = fcm(bonds,numClust,options);
[~, fidx] = max(U);
fidx = fidx';


P = array2table([A(:,n) fidx]);
writetable(P, [datadir 'versus_cmeans_' fileruns(r).name] );
display([datadir 'versus_cmeans_' fileruns(r).name]);
...
\end{lstlisting}
\subsection{Calcolo delle prestazioni}
Ogni algoritmo di clustering, per tutte le combinazioni di finestra ed overlap, fornisce un vettore di etichette. Questo rappresenta la divisione in cluster effettuata dall'algoritmo, i quali andranno confrontati con le etichette reali del dataset. Per fare questo, procediamo al calcolo di metriche quali, in ordine, accuratezza, precisione, sensitivita' e F1-measure. Per quanto riguarda la precisione e la sensitivita', esse sarebbero relative ad ogni classe, ma poiche' ci interessa indovinare la maggior parte delle volte tutte e 3 le classi, quelle che saranno riportate sono la media della precisione e sensitivita' delle classi. Il codice implementativo e' fornito in \ref{rate}

\begin{lstlisting}[style=Matlab-editor,frame=single, caption=Calcolo delle prestazioni, label=rate]  % Start your code-block
...
% Per tutti i file del paziente $isubject
for r = 1:length(fileruns)

% Etichette del dataset
filename = [datadir fileruns(r).name];
T1 = readtable(filename);
[m1,n1] = size(T1);
A1 = table2array(T1(:,n1));

% Etichette cluster
filename2 = [datadir fileruns2(r).name];
T2 = readtable(filename2);
[m2,n2] = size(T2);
A2 = table2array(T2(:,2));
% Etichette cluster e dataset
D = [A2 A1];

% Matrice di confusione
[C,order] = confusionmat(D(:,2),D(:,1));
accuracy = c_accuracy(C);
precision = c_precision(C);
recall = c_recall(C);
F1measure = c_F1measure(precision,recall);

B = [accuracy precision recall F1measure];
E = [E B];

end
Q = [Q ; [e E]];
e = [e [0 0 0 0]];
...
function accuracy = c_accuracy(C)
giusti = sum(diag(C));
totale = sum(sum(C));
accuracy = giusti/totale;
end

function precision = c_precision(C)
precision1 = C(1,1)/(C(1,1)+C(2,1)+C(3,1));
precision2 = C(2,2)/(C(1,2)+C(2,2)+C(3,2));
precision3 = C(3,3)/(C(1,3)+C(2,3)+C(3,3));
precision = mean([precision1;precision2;precision3]);
end

function recall = c_recall(C)
recall1 = C(1,1)/(C(1,1)+C(1,2)+C(1,3));
recall2 = C(2,2)/(C(2,1)+C(2,2)+C(2,3));
recall3 = C(3,3)/(C(3,1)+C(3,2)+C(3,3));
recall = mean([recall1;recall2;recall3]);
end

function F1measure = c_F1measure(precision,recall)
F1measure = 2*precision*recall/(precision+recall);
end
\end{lstlisting}
Viene creata dunque una tabella che riassume, per ogni possibile intervallo, le prestazioni dell'algoritmo, come si vede in \ref{ratecluster1}, \ref{ratecluster2}, \ref{ratecluster3}.
\begin{table}[htp]
	\centering
	\caption{Esempio di tabelle delle prestazioni del clustering}
	\label{ratecluster1}
	\resizebox{\textwidth}{!}{%
	\begin{tabular}{|l|l|l|l|l|l|l|l|l|l|l|l|l|}
		\hline
		& \multicolumn{4}{c|}{Secondi: 1} & \multicolumn{4}{c|}{Secondi: 1.5} & \multicolumn{4}{c|}{Secondi: 2} \\ \hline
		Ov 0.5 & 0,7 & 0,6 & 0,3 & 0,4 & 0,8 & 0,3 & 0,3 & 0,3 & 0,91 & 0,5 & 0,5 & 0,5 \\ \hline
		Ov 1 & 0 & 0 & 0 & 0 & 0,7 & 0,3 & 0,3 & 0,3 & 0,8 & 0,6 & 0,5 & 0,55 \\ \hline
		Ov 1.5 & 0 & 0 & 0 & 0 & 0 & 0 & 0 & 0 & 0,91 & 0,72 & 0,6 & 0,6 \\ \hline
	\end{tabular}%
}
\end{table}
\begin{table}[htp]
	\centering
	\caption{Esempio di tabelle delle prestazioni del clustering}
	\label{ratecluster2}
	\resizebox{\textwidth}{!}{%
	\begin{tabular}{|l|l|l|l|l|l|l|l|l|l|l|l|l|}
		\hline
		& \multicolumn{4}{c|}{Secondi: 2.5} & \multicolumn{4}{c|}{Secondi: 3} & \multicolumn{4}{c|}{Secondi: 3.5} \\ \hline
		Ov 0.5 & 0,7 & 0,3 & 0,3 & 0,3 & 0,4 & 0,1 & 0,2 & 0,17 & 0,46 & 0,32 & 0,4 & 0,35 \\ \hline
		Ov 1 & 0,05 & 0,3 & 0,2 & 0,3 & 0,46 & 0,3 & 0,4 & 0,38 & 0,46 & 0,32 & 0,4 & 0,35 \\ \hline
		Ov 1.5 & 0,75 & 0,2 & 0,2 & 0,2 & 0,8 & 0,3 & 0,3 & 0,3 & 0,46 & 0,32 & 0,4 & 0,35 \\ \hline
		Ov 2 & 0,8 & 0,3 & 0,3 & 0,3 & 0,6 & 0,3 & 0,1 & 0,21 & 0,46 & 0,32 & 0,4 & 0,35 \\ \hline
		Ov2.5 & 0 & 0 & 0 & 0 & 0,81 & 0,26 & 0,24 & 0,25 & 0,7 & 0,3 & 0,2 & 0,26 \\ \hline
		Ov 3 & 0 & 0 & 0 & 0 & 0 & 0 & 0 & 0 & 0,6 & 0,3 & 0,31 & 0,31 \\ \hline
	\end{tabular}%
}
\end{table}
\begin{table}[htp]
	\centering
	\caption{Esempio di tabelle delle prestazioni del clustering}
	\label{ratecluster3}
	\resizebox{\textwidth}{!}{%
	\begin{tabular}{|l|l|l|l|l|l|l|l|l|l|l|l|l|}
		\hline
		& \multicolumn{4}{c|}{Secondi: 4} & \multicolumn{4}{c|}{Secondi: 4.5} & \multicolumn{4}{c|}{Secondi: 5} \\ \hline
		Ov 0.5 & 0,8 & 0,36 & 0,35 & 0,35 & 0,81 & 0,32 & 0,34 & 0,33 & 0,75 & 0,24 & 0,26 & 0,25 \\ \hline
		Ov 1 & 0,76 & 0,21 & 0,28 & 0,26 & 0,79 & 0,32 & 0,34 & 0,31 & 0,79 & 0,3 & 0,3 & 0,3 \\ \hline
		Ov 1.5 & 0,46 & 0,36 & 0,33 & 0,35 & 0,8 & 0,36 & 0,35 & 0,35 & 0,59 & 0,24 & 0,33 & 0,29 \\ \hline
		Ov 2 & 0,41 & 0,33 & 0,44 & 0,39 & 0,76 & 0,21 & 0,28 & 0,26 & 0,26 & 0,79 & 0,32 & 0,34 \\ \hline
		Ov2.5 & 0,59 & 0,24 & 0,33 & 0,29 & 0,46 & 0,36 & 0,33 & 0,35 & 0,35 & 0,8 & 0,36 & 0,35 \\ \hline
		Ov 3 & 0,44 & 0,11 & 0,12 & 0,11 & 0,41 & 0,33 & 0,44 & 0,39 & 0,8 & 0,36 & 0,35 & 0,35 \\ \hline
		Ov 3.5 & 0,78 & 0,23 & 0,33 & 0,28 & 0,75 & 0,24 & 0,26 & 0,25 & 0,76 & 0,21 & 0,28 & 0,26 \\ \hline
		Ov 4 & 0 & 0 & 0 & 0 & 0,79 & 0,3 & 0,3 & 0,3 & 0,75 & 0,24 & 0,26 & 0,25 \\ \hline
		Ov 4.5 & 0 & 0 & 0 & 0 & 0 & 0 & 0 & 0 & 0,79 & 0,3 & 0,3 & 0,3 \\ \hline
	\end{tabular}%
	}
\end{table}
\subsection{Risultati}
Dopo che sono state generate tutte le tabelle per tutti i pazienti in cui vengono espressi i valori di accuratezza, precisione, sensitivita' e F1-measure, siamo interessati all'intervallo migliore, ossia quello che ci fornisce i valori maggiori delle metriche suddette. Per fare questo, analizziamo ogni tabella e, per ogni algoritmo, salviamo l'intervallo ed i valori che presentano le metriche migliori, come nel caso delle tabelle \ref{ratecmeans}, \ref{ratekmeans}, \ref{rateneuralnetwork}.\\
Ogni tabella descrive:
\begin{itemize}
	\item l'algoritmo di clustering che e' stato utilizzato;
	\item il paziente del test;
	\item i valori delle metriche migliori trovate;
	\item la combinazione overlap-secondi in cui risultano tali metriche.
\end{itemize}

% Please add the following required packages to your document preamble:
% \usepackage{graphicx}
\begin{table}[]
	\centering
	\caption{Rate Pazienti per l'algoritmo di clustering C-means}
	\label{ratecmeans}
	\resizebox{\textwidth}{!}{%
		\begin{tabular}{|l|l|l|l|l|l|l|}
			\hline
			\textbf{}         & \multicolumn{6}{c|}{\textbf{CMEANS}}                                                                              \\ \hline
			\textbf{Paziente} & \textbf{Accuratezza} & \textbf{Precisione} & \textbf{Recall} & \textbf{F1} & \textbf{Overlap} & \textbf{Finestra} \\ \hline
			\textbf{S01}      & 0,92                 & 0,31                & 0,33            & 0,32        & 1                & 2                 \\ \hline
			\textbf{S02}      & 0,44                 & 0,46                & 0,63            & 0,53        & 1                & 2                 \\ \hline
			\textbf{S03}      & 0,7                  & 0,51                & 0,57            & 0,54        & 1                & 2                 \\ \hline
			\textbf{S04}      & 0,7                  & 0,33                & 0               & 0           & 0,5                & 1               \\ \hline
			\textbf{S05}      & 0,53                 & 0,49                & 0,46            & 0,47        & 1                & 1,5               \\ \hline
			\textbf{S06}      & 0,57                 & 0,36                & 0,32            & 0,34        & 1                & 1,5               \\ \hline
			\textbf{S07}      & 0,48                 & 0,37                & 0,43            & 0,4         & 1                & 1,5               \\ \hline
			\textbf{S08}      & 0,41                 & 0,39                & 0,43            & 0,41        & 1,5              & 2                 \\ \hline
			\textbf{S09}      & 0,59                 & 0,35                & 0,37            & 0,36        & 1                & 1,5               \\ \hline
			\textbf{S10}      & 0,73                 & 0,33                & 0               & 0           & 0,5                & 1               \\ \hline
		\end{tabular}%
	}
\end{table}

% Please add the following required packages to your document preamble:
% \usepackage{graphicx}
\begin{table}[]
	\centering
	\caption{Rate Pazienti per l'algoritmo di clustering K-means}
	\label{ratekmeans}
	\resizebox{\textwidth}{!}{%
		\begin{tabular}{|l|l|l|l|l|l|l|}
			\hline
			\textbf{Tutti}    & \multicolumn{6}{c|}{\textbf{KMEANS}}                                                                              \\ \hline
			\textbf{Paziente} & \textbf{Accuratezza} & \textbf{Precisione} & \textbf{Recall} & \textbf{F1} & \textbf{Overlap} & \textbf{Finestra} \\ \hline
			\textbf{S01}      & 0,92                 & 0,72                & 0,34            & 0,46        & 1,5              & 2                 \\ \hline
			\textbf{S02}      & 0,82                 & 0,61                & 0,34            & 0,44        & 1                & 2                 \\ \hline
			\textbf{S03}      & 0,75                 & 0,78                & 0,54            & 0,64        & 1,5              & 2                 \\ \hline
			\textbf{S04}      & 1                    & 0,33                & 0               & 0           & 0,5                & 1               \\ \hline
			\textbf{S05}      & 0,67                 & 0,56                & 0,33            & 0,42        & 1                & 2                 \\ \hline
			\textbf{S06}      & 0,91                 & 0,3                 & 0,33            & 0,32        & 1                & 1,5               \\ \hline
			\textbf{S07}      & 0,92                 & 0,31                & 0,33            & 0,32        & 1                & 2                 \\ \hline
			\textbf{S08}      & 0,7                  & 0,57                & 0,34            & 0,42        & 1                & 1,5               \\ \hline
			\textbf{S09}      & 0,81                 & 0,6                 & 0,34            & 0,43        & 1                & 2                 \\ \hline
			\textbf{S10}      & 1                    & 0,33                & 0               & 0           & 0,5                & 1,5               \\ \hline
		\end{tabular}%
	}
\end{table}

% Please add the following required packages to your document preamble:
% \usepackage{graphicx}
\begin{table}[]
	\centering
	\caption{Rate Pazienti per l'algoritmo di clustering Self-Organizing Map}
	\label{rateneuralnetwork}
	\resizebox{\textwidth}{!}{%
		\begin{tabular}{|l|l|l|l|l|l|l|}
			\hline
			\textbf{Tutti}    & \multicolumn{6}{c|}{\textbf{NEURAL NETWORK}}                                                                      \\ \hline
			\textbf{Paziente} & \textbf{Accuratezza} & \textbf{Precisione} & \textbf{Recall} & \textbf{F1} & \textbf{Overlap} & \textbf{Finestra} \\ \hline
			\textbf{S01}      & 0,92                 & 0,31                & 0,33            & 0,32        & 1                & 1,5               \\ \hline
			\textbf{S02}      & 0,81                 & 0,44                & 0,34            & 0,38        & 1                & 1,5               \\ \hline
			\textbf{S03}      & 0,75                 & 0,78                & 0,54            & 0,63        & 1                & 2                 \\ \hline
			\textbf{S04}      & 0,64                 & 0,33                & 0               & 0           & 0,5                & 1                 \\ \hline
			\textbf{S05}      & 0,58                 & 0,42                & 0,46            & 0,44        & 1                & 2                 \\ \hline
			\textbf{S06}      & 0,57                 & 0,32                & 0,31            & 0,31        & 1                & 1,5               \\ \hline
			\textbf{S07}      & 0,63                 & 0,3                 & 0,26            & 0,28        & 1                & 2                 \\ \hline
			\textbf{S08}      & 0,56                 & 0,69                & 0,38            & 0,49        & 1,5              & 2                 \\ \hline
			\textbf{S09}      & 0,57                 & 0,37                & 0,41            & 0,39        & 1                & 2                 \\ \hline
			\textbf{S10}      & 1                    & 0,33                & 0               & 0           & 0,5                & 1               \\ \hline
		\end{tabular}%
	}
\end{table}

Osservando le tabelle, si nota come il valore di overlap e dimensione della finestra che piu' frequentemente mi portano ad avere metriche migliori sono:
\begin{itemize}
	\item 1 secondo di sovrapposizione e 2 secondi di finestra;
	\item 1 secondo di sovrapposizione e 1,5 secondi di finestra.
\end{itemize}
Il nostro studio degli intervalli quindi ci porta a concludere che il miglior modo con cui si possono dividere i dati in finestre per calcolare le feature e' scegliere 1 secondo di sovrapposizione tra le finestre e 1,5 o 2 secondi per quanto riguarda la lunghezza della finestra temporale.

\section{Etichettatura dei Dati}
Nella sezione precedente, lo studio degli intervalli ci ha portato a scegliere, come dimensioni delle finestre temporali, 1,5 o 2 secondi con 1 secondo di overlap. A questo punto, usando tali valori, ritestiamo per ogni paziente gli algoritmi di clustering ed andiamo ad analizzare direttamente le matrici di confusione, le quali ci diranno esattamente come etichettano tali algoritmi a confronto con le etichette reali.\\

%% !TEX root = ../main.tex

% ---------------------------------------------------------------
% Software Implamentation of metodology
% ---------------------------------------------------------------
\chapter[Implementazione Software]{Implementazione software della metodologia proposta}\label{chap6:SW_IMP_MET}
\section{Implementazione Feature Statistiche}


\section{Implementazione Feature Dinamiche}


\section{Implementazione Linear Discriminant Analysis}
%% !TEX root = ../main.tex

% ---------------------------------------------------------------
% Case Studies
% ---------------------------------------------------------------

\chapter{Case Studies}\label{chap7:CS}

%% !TEX root = ../main.tex

\chapter{Experimental Results}\label{chap8:results}
The proposed methodology has been implemented in a tool that automatically generates the self-adaptive wrapper for connecting a cycle-accurate TLM SystemC target to a generic (not necessarily cycle-accurate) TLM SystemC initiator.
Its effectiveness has been evaluated by generating the self-adaptive wrappers for the IPs reported in Table~\ref{TAB:DESIGN}, whose RTL versions have been retrieved from the Open Core Library. Cycle-accurate TLM SystemC models of the considered IPs have been obtained by using HIFSuite-A2T. 
%abstracting the RTL descriptions as reported in Section~\ref{SEC:F_ABSTRACTION}.

Table~\ref{TAB:DESIGN} reports first information related to the RTL version of the IPs, i.e., the number of assertions defined to describe their cycle-accurate communication protocol ($|A|$), the lines of code ($\#lines$), the number of processes ($\#proc$) and the number of input/output ports ($\#I/O$). Then, it reports the number of fields included in the payload of the cycle accurate TLM target ($\#P^T$), which corresponds to $\#I/O-1$ (only clock is removed), and the number of fields included in the payload of the initiator ($\#P^I$), which is lower that $\#P^T$, since the initiator implements a more abstracted loosely timed protocol, where some ports included in the RTL IPs have been removed.

Table~\ref{TAB:RES} reports the simulation times, in seconds, achieved by stimulating each IP with a testbench that generates a sequence of $10^5$ input stimuli. 
The simulation has been performed by composing the testbench, which represents the initiator, and the IP, which represents the target, in different ways and at different abstraction levels.
Column \textit{RTL-TLM (trans)} shows the worst case, where the RTL version of the IP is connected to a loosely timed TLM testbench by means of an ad hoc transactor. This is the simplest solution to reuse an already existing RTL IP inside a TLM design, when a corresponding TLM version of the IP is not available.
On the contrary, Column \textit{LT-LT (no trans)} shows the ideal case, where both the testbench and the target IP implement a perfectly matching, loosely timed TLM protocol. This happens, generally, when the TLM version of the IP is manually implemented. In this case, pin-accurate ports that implement low-level details of the RTL protocol are not included in the payload, thus  the maximum grade of abstraction is reached. In this case, there is no need of generating the transactor between the testbench and the IP. 
Since, manual implementation of a TLM model is time-consuming and error-prone, the fastest and the most safe solution, when an RTL IP of the target design is available, consists of using tools that automatically abstract the IP towards TLM.
Unfortunately, commercial tools that automatically perform RTL-to-TLM abstraction generate cycle-accurate TLM models, since they are not able to automatically abstract low-level details related to the pin-accurate RTL protocol. In such cases, only a TLM cycle-accurate simulation is possible by directly connecting the abstracted IP with a cycle-accurate initiator, which is represented by Column \textit{CA-CA (no trans)}. However, if the initiator implements a more abstracted protocol than a cycle-accurate one, a wrapper is required to allow its communication with the cycle-accurate abstracted IP. This case is showed in Column \textit{LT-CA  (wrapper)}, where we adopted the self-adaptive wrappers that have been automatically generated according to the proposed methodology .

From Table~\ref{TAB:RES} we first observe that for all the benchmarks, but UART, the mixed RTL-TLM simulation via the transactor is orders of magnitude slower than the other simulation schemas. This confirms that the adoption of the transactor generally slows down the overall simulation at the speed of the RTL models, thus making inconvenient the reuse of RTL IPs through transactors. 
This drawback is not particularly evident for the UART, because its RTL model is a structural (not behavioural) description. In such a case, the automatic abstraction of the UART's functionality, via HIFSuite-A2T, generates a TLM version with a low speed-up with respect to the original RTL model. However, this does not depend on the methodology we propose in \chapterref{chap5:Automatic} and \chapterref{chap6:SW_IMP_MET} for the generation of the wrapper, which is independent form the way we abstract the functionality.
Indeed, the efficiency of our self-adaptive wrapper is confirmed by comparing Columns \textit{LT}-LT, \textit{CA-CA} and \textit{LT-CA} for all the benchmarks. The time required by the mixed LT-CA simulation with the use of our wrapper introduces a small overhead in the simulation in comparison with the  LT-LT simulation, and it is faster than the  CA-CA simulation.
This highlights the efficiency of the overall methodology foreseen in \ref{fig:wrapper-gen}, which combines the use of an automatic procedure for abstracting the IP functionality of an RTL IP towards a cycle-accurate TLM model, and the automatic generation of the proposed self-adaptive wrapper to allow the communication between the cycle-accurate TLM model and more abstracted TLM components.



\begin{figure}[h!]
\centering
%\scriptsize
\begin{tabular}{ l || c |c |c |c ||c| c|| }
\multicolumn{1}{l||}{} & 
\multicolumn{4}{c||}{RTL} & 
\multicolumn{2}{c||}{TLM} \\
%\cline{2-7}
$Design$	&	
$|A|$		&	
$\#lines$	&	
$\#proc$	&
$\#I/O$	&
 $\#P^T$	&	
 $\#P^I$\\
\hline
\hline
AES			&	7		&	1778	&	30 	&	8 	& 	7 	&	5	\\
Camellia	&	12		&	955		&	49	&	12	&	11	&	6	\\
DES56 		&	10		&	1054	&	6	&	10	&	9	&	4	\\
UART		&	9		&	3262	&	60	&	23	&	22	&	20\\
%\hline
\end{tabular}
\vspace{0.1cm}
\caption{Characteristics of benchmarks.}
%\vspace{-0.2cm}
\label{TAB:DESIGN}
\end{figure}

\begin{figure}[h!]
\centering
%\scriptsize
\begin{tabular}{ l || c |c |c |c || }
\multirow{2}{*}{$Design$} &
RTL-TLM &
LT-LT &
CA-CA &
LT-CA					\\
						&
$(trans)$ 			&
$(no~trans)$ 	&
$(no~trans)$ 	&
$(wrapper)$				\\
\hline
\hline
AES			&	26495	&844		&904	&			911	\\
Camellia	&	7883	&9			&11		&			10	\\
DES56 		&	410		&45			&54		&			52	\\
UART		&	4300	&3770		&3960	&			3870\\
%\hline
\end{tabular}
\vspace{0.1cm}
\caption{Simulation times (in seconds).}
\vspace{-0.3cm}
\label{TAB:RES}
\end{figure}

% !TEX root = ../main.tex

\chapter{Conclusione e Lavori Futuri}\label{chap9:Concl}
This thesis proposes a methodology 

\section{Lavori Futuri}


\begin{comment}
% Appendices
\appendix
\setChapters{Appendix}
\renewcommand{\chaptermark}[1]%
{\markboth{\textbf{\appendixname\ \thechapter.\ #1}}{}}
\fancyhead[LO, RE]{\leftmark}
\input{appendices/appendix1.tex}
\end{comment}
% ---------------------------------------------------------------
% Backmatter 
% ---------------------------------------------------------------
\backmatter

% Bibliography
\renewcommand{\chaptermark}[1]%
{\markboth{\textbf{#1}}{}}
\fancyhead[LO, RE]{\leftmark}
\chapter{Bibliography}
%\printbibliography[heading=subbibliography, title={Book References}, filter=books]
%\printbibliography[heading=subbibliography, title={Article References}, filter=papers]
%\printbibliography[heading=subbibliography, title={Other References}, filter=others]
\printbibliography[heading=subbibliography]
% List of acronyms
\chapter*{Acronyms}
\begin{acronym}[AAAAAAAAA]%Longest
\acro{FI}{Freezing Index}
\acro{MEMS}{MicroElectroMechanical Systems}
\acro{PSD}{Power Spectral Density}
\acro{FB}{Freezing Band}
\acro{WB}{Walking Band}
\acro{EMG}{Elettromiografia}
\acro{EEG}{Elettroencefalogramma}
\acro{FTH}{Freezing Threshold}
\acro{FP}{False Positive}
\acro{PI}{Power Index}
\acro{PT}{Power Threshold}
\acro{MBFA}{Moore-Bächlin FoG Algorithm}
\acro{PD}{Parkinson's disease}
\end{acronym}

\end{document}